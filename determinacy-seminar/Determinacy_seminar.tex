\newcommand{\mytitle}{Determinacy -- seminar}
% Document definition
\documentclass[a4paper,10pt]{article}
% Packages
% General packages
  \usepackage[T1]{fontenc}
  \usepackage[latin1]{inputenc} % Danish language support
  \usepackage{sectsty} % Makes it possible to manipulate fonts
  \usepackage[light]{antpolt} % Provides awesome font
  \usepackage{hyperref} % Provides \url and clickable links in the pdf
  %\usepackage{cleveref}	% Provides \cref and \Cref for including text before references
  \usepackage{graphicx} % Provides the \includegraphics[]{} command
  \usepackage{comment} % Provides comment-environment for multi-line commenting
  \usepackage{twoopt} % Allows adding commands with two optional arguments
  \usepackage{setspace} % Provides onehalfspacing environment
  \usepackage{enumitem} % Provides control of spacing in lists
  \usepackage{amsbsy} % Provides \boldsymbol command
  \usepackage{apacite} % Provides bibliography style
  \usepackage[sectionbib]{natbib} % Provides \citep and \citeyearpar commands
  \usepackage{todonotes} % Provides to do command
  \usepackage{wrapfig} % For wrapping figures
  %\usepackage{faktor} % Allows for quotients like A/B **PRODUCES ERROR ON DAN'S OFFICE COMPUTER**
  \usepackage{makeidx} % Enables index
  \usepackage{tocloft} % Allows changing spacing in TOC
	
  % Math packages
  \usepackage{amsmath, amssymb, amsfonts} % Math symbolic jargon
  \usepackage{amsthm} % Theorem environment
  \usepackage{mathrsfs}	% Provides the \mathscr{} curly font
  \usepackage{stmaryrd}	% Provides the \lightning symbol and semantics-brackets, among others
  \usepackage{mathabx} % Provides the \dotdiv symbol	
  \usepackage{proof} % Natural deduction with \infer
  \usepackage{tikz} % Awesome diagrams
    \usetikzlibrary{cd} % Commutative diagrams 
    \usetikzlibrary{matrix,arrows} % Matrices and arrows for style points
    \usetikzlibrary{decorations.pathreplacing,calc,arrows.meta} % For making tree cds
  \usepackage[all]{xy} % Provides xymatrix environment for diagrams
		\newdir{|>}{-<0pt,5pt>{\blacktriangleup}}
  \usepackage{pigpen} % Provides pullback/pushout symbols in diagrams
  \usepackage{xfrac} % Provides \sfrac command for diagonal fraction
	%\usepackage{undertilde} % Provides \utilde command
		
  % Page layout
  %\usepackage{fullpage} % Reduces margins
  \usepackage{wallpaper} % Adds the \ThisLRCornerWallPaper command
  \usepackage{fancyhdr}	% Provides headers
  %\usepackage[compact,small]{titlesec} % Makes titles a bit smaller with shorter proceeding space
  \usepackage{changepage} % Provides adjustwidth environment for claims

  %\usepackage{xargs} % better handling of function/command arguments **PRODUCES ERROR ON DAN'S OFFICE COMPUTER**

  % Todonotes
  \usepackage{todonotes} % Enables the \todo command
  \newcommand{\todocomment}[2][]{\todo[linecolor=blue,backgroundcolor=blue!25,bordercolor=blue,#1]{#2}}

% Tree diagrams
\tikzset{
  tree/.style 2 args={
    decorate,
    decoration={
      show path construction,
      lineto code={
        \draw[dotted,-] (\tikzinputsegmentfirst) --($(\tikzinputsegmentfirst)!.5!(\tikzinputsegmentlast)$);
        \draw[-{Latex}] ($(\tikzinputsegmentfirst)!.5!(\tikzinputsegmentlast)$) --(\tikzinputsegmentlast) node [midway,right] {\small{$#1$}};
        \draw[-] (\tikzinputsegmentfirst) --++ (105:0.65cm);
        \draw[-] (\tikzinputsegmentfirst) --++ (75:0.65cm) node [midway, right] {\small{$#2$}};
      }
    }
  }
}

\tikzset{
  treeplain/.style 2 args={
    decorate,
    decoration={
      show path construction,
      lineto code={
        \draw[dotted,-] (\tikzinputsegmentfirst) --($(\tikzinputsegmentfirst)!.5!(\tikzinputsegmentlast)$);
        \draw[-] ($(\tikzinputsegmentfirst)!.5!(\tikzinputsegmentlast)$) --(\tikzinputsegmentlast) node [midway,right] {\small{$#1$}};
        \draw[-] (\tikzinputsegmentfirst) --++ (105:0.65cm);
        \draw[-] (\tikzinputsegmentfirst) --++ (75:0.65cm) node [midway, right] {\small{$#2$}};
      }
    }
  }
}

% Make \emph boldface
\let\emph\relax % there's no \RedeclareTextFontCommand
\DeclareTextFontCommand{\emph}{\bfseries}

% Prevent widows and orphans
\widowpenalty = 10000
\clubpenalty = 10000

% List spacing
\setlist{nolistsep}

% No indent
\setlength{\parindent}{0ex}

% Line spacing (1.3 = one and a half spacing)
\linespread{1.3}

% Theorem environment
\newtheoremstyle{scthmstyle} % Name
	{15pt} % Space above
	{15pt} % Space below
	{\itshape} % Body font
	{} % Indent amount
	{\bfseries\scshape} % Theorem head font
	{.} % Punctuation after theorem head
	{.5em} % Space after theorem head
	{} % Theorem head spec (can be left empty, meaning ënormalí)

\newtheoremstyle{scdefstyle} % Name
	{15pt} % Space above
	{15pt} % Space below
	{\normalfont} % Body font
	{} % Indent amount
	{\bfseries\scshape} % Theorem head font
	{.} % Punctuation after theorem head
	{.5em} % Space after theorem head
	{} % Theorem head spec (can be left empty, meaning ënormalí)

\newtheoremstyle{scremstyle} % Name
	{15pt} % Space above
	{15pt} % Space below
	{\normalfont} % Body font
	{} % Indent amount
	{\itshape} % Theorem head font
	{.} % Punctuation after theorem head
	{.5em} % Space after theorem head
	{} % Theorem head spec (can be left empty, meaning ënormalí)

\newtheoremstyle{scclaistyle} % Name
	{15pt} % Space above
	{15pt} % Space below
	{\normalfont} % Body font
	{0.5cm} % Indent amount
	{\itshape} % Theorem head font
	{.} % Punctuation after theorem head
	{.5em} % Space after theorem head
	{} % Theorem head spec (can be left empty, meaning ënormalí)

\theoremstyle{scthmstyle}
\newtheorem{theorem}{Theorem}[section]
\newtheorem{proposition}[theorem]{Proposition}
\newtheorem{lemma}[theorem]{Lemma}
\newtheorem{corollary}[theorem]{Corollary}
\theoremstyle{scdefstyle}
\newtheorem{definition}[theorem]{Definition}
\newtheorem{convention}[theorem]{Convention}
\newtheorem{example}[theorem]{Example}
\newtheorem{question}[theorem]{Question}
\theoremstyle{scremstyle}
\newtheorem{remark}[theorem]{Remark}
\theoremstyle{scclaistyle}
\newtheorem{claim}{Claim}[theorem]

% User-defined commands
  % General things	
  \newcommand{\eq}[1]{\begin{align*} #1 \end{align*}}
  \newcommand{\eqq}[1]{\begin{align*} #1\\ \end{align*}}
  \newcommand{\pix}[2][1]{\begin{center}\includegraphics[scale=#1]{#2}\\\end{center}}
  \newcommand{\cd}[1]{\eq{\xymatrix{#1}}}
  \renewcommand{\labelenumi}{(\roman{enumi}) } % Using roman numerals in lists
  \renewcommand{\b}[1]{{\bf #1}}
  \renewcommand{\abstract}[1]{\begin{quote}{\footnotesize\textsc{Abstract.} #1}\\\end{quote}}

  % Theorem environments
  \newcommandtwoopt{\theo}[3][][]{
    \begin{theorem}[#1]\label{#2}
      #3
    \end{theorem}}
  \newcommandtwoopt{\prop}[3][][]{
    \begin{proposition}[#1]\label{#2}
      #3
    \end{proposition}}
  \newcommandtwoopt{\lemm}[3][][]{
    \begin{lemma}[#1]\label{#2}
      #3
    \end{lemma}}
  \newcommandtwoopt{\coro}[3][][]{
    \begin{corollary}[#1]\label{#2}
      #3
    \end{corollary}}
  \newcommandtwoopt{\defi}[3][][]{
    \begin{definition}[#1]\label{#2}
      #3$\hfill\dashv$
    \end{definition}}
  \newcommandtwoopt{\exam}[3][][]{
    \begin{example}[#1]\label{#2}
      #3
    \end{example}}
  \newcommandtwoopt{\ques}[3][][]{
    \begin{question}[#1]\label{#2}
      #3
    \end{question}}
  \newcommandtwoopt{\rema}[3][][]{
    \begin{remark}[#1]\label{#2}
      #3
    \end{remark}}

  \newcommandtwoopt{\qtheo}[3][][]{
    \begin{theorem}[#1]\label{#2}
      #3$\hfill\dashv$
    \end{theorem}}
  \newcommandtwoopt{\qprop}[3][][]{
    \begin{proposition}[#1]\label{#2}
      #3$\hfill\dashv$
    \end{proposition}}
  \newcommandtwoopt{\qlemm}[3][][]{
    \begin{lemma}[#1]\label{#2}
      #3$\hfill\dashv$
    \end{lemma}}
  \newcommandtwoopt{\qcoro}[3][][]{
    \begin{corollary}[#1]\label{#2}
      #3$\hfill\dashv$
    \end{corollary}}

  \newcommandtwoopt{\defin}[3][][]{
    \begin{definition}[#1]\label{#2}
      #3
    \end{definition}}

  \newcommand{\proofretard}{\textsc{Proof.} }
  \renewcommand{\proof}[1]{\textsc{Proof.} #1$\qed$\\}
  \newcommand{\clai}[2][]{\begin{claim}[#1]#2\end{claim}}
  \newcommand{\cproof}[1]{
    \begin{adjustwidth}{0.5cm}{0pt}
      \textsc{Proof of claim.} #1$\hfill\dashv$\\
    \end{adjustwidth}}
  \renewcommand{\qed}{\hfill\blacksquare}
  \newcommand{\qedeq}{\tag*{$\blacksquare$}}

  % Declared operators
  \DeclareMathOperator{\dirlim}{dirlim}
  \DeclareMathOperator{\iterates}{I}
  \DeclareMathOperator{\code}{Code}
  \DeclareMathOperator{\piterates}{pI}
  \DeclareMathOperator{\blowups}{B}
  \DeclareMathOperator{\pblowups}{pB}
  \DeclareMathOperator{\lhod}{\triangleleft_{\mathrm{HOD}}}
  \DeclareMathOperator{\lehod}{\trianglelefteq_{\mathrm{HOD}}}
  \DeclareMathOperator{\ledj}{\le_{\mathrm{DJ}}}
  \DeclareMathOperator{\nledj}{\not{\le}_{\mathrm{DJ}}}
  \DeclareMathOperator{\ldj}{<_{\mathrm{DJ}}}
  \DeclareMathOperator{\di}{\textsf{DI}}
  \DeclareMathOperator{\hc}{\textsf{HC}}
  \DeclareMathOperator{\M}{\mathcal M}
  \DeclareMathOperator{\N}{\mathcal N}
  \DeclareMathOperator{\K}{\mathcal K}
  \DeclareMathOperator{\F}{\mathcal F}
  \DeclareMathOperator{\Q}{\mathcal Q}
  \DeclareMathOperator{\W}{\mathcal W}
  \DeclareMathOperator{\V}{\mathcal V}
  \DeclareMathOperator{\T}{\mathcal T}
  \DeclareMathOperator{\U}{\mathcal U}
  \DeclareMathOperator{\B}{\mathcal B}
  \DeclareMathOperator{\R}{\mathcal R}
  \DeclareMathOperator{\D}{\mathcal D}
  \DeclareMathOperator{\G}{\mathcal G}
  \DeclareMathOperator{\C}{\mathcal C}
  \DeclareMathOperator{\A}{\mathcal A}
  \DeclareMathOperator{\h}{\mathcal H}
  \DeclareMathOperator{\core}{\mathfrak C}
  \DeclareMathOperator{\mc}{\textsf{MC}}
  \DeclareMathOperator{\refl}{Refl}
  \DeclareMathOperator{\pp}{pp}
  \DeclareMathOperator{\on}{On}
  \DeclareMathOperator{\rk}{rk}
  \DeclareMathOperator{\otp}{otp}
  \DeclareMathOperator{\pr}{pr}
  \DeclareMathOperator{\lp}{Lp}
  \DeclareMathOperator{\In}{in}
  \DeclareMathOperator{\sgn}{sgn}
  \DeclareMathOperator{\lcm}{lcm}
  \DeclareMathOperator{\ran}{ran}
  \DeclareMathOperator{\cod}{cod}
  \DeclareMathOperator{\dom}{dom}	
  \DeclareMathOperator{\cond}{cond}
  \DeclareMathOperator{\con}{Con}
  \DeclareMathOperator{\rank}{rank}
  \DeclareMathOperator{\gal}{Gal}
  \DeclareMathOperator{\cov}{Cov}
  \DeclareMathOperator{\im}{im}
  \DeclareMathOperator{\sub}{Sub}
  \DeclareMathOperator{\diam}{diam}
  \DeclareMathOperator{\hod}{HOD}
  \DeclareMathOperator{\od}{OD}
  \DeclareMathOperator{\codom}{codom}
  \DeclareMathOperator{\Det}{Det}
  \DeclareMathOperator{\len}{len}
  \DeclareMathOperator{\ma}{MA}
  \DeclareMathOperator{\id}{id}
  \DeclareMathOperator{\sing}{Sing}
  \DeclareMathOperator{\pred}{pred}
  \DeclareMathOperator{\cl}{cl}
  \DeclareMathOperator{\Int}{int}
  \DeclareMathOperator{\ob}{Ob}
  \DeclareMathOperator{\mor}{Mor}
  \DeclareMathOperator{\sh}{Sh}
  \DeclareMathOperator{\ot}{ot}
  \DeclareMathOperator{\ult}{Ult}
  \DeclareMathOperator{\sg}{sg}
  \DeclareMathOperator{\env}{Env}
  \DeclareMathOperator{\tor}{Tor}
  \DeclareMathOperator{\ext}{Ext}
  \DeclareMathOperator{\comp}{\textsf{Comp}}
  \DeclareMathOperator{\card}{card}
  \DeclareMathOperator{\cf}{cf}
  \DeclareMathOperator{\cof}{cof}
  \DeclareMathOperator{\cc}{\text{\textasciicircum}}
  \DeclareMathOperator{\sk}{sk}
  \DeclareMathOperator{\crit}{crit}
  \DeclareMathOperator{\cls}{cls}
  \DeclareMathOperator{\pd}{pd}
  \DeclareMathOperator{\ev}{ev}
  \DeclareMathOperator{\wo}{WO}
  \DeclareMathOperator{\wfp}{wfp}
  \DeclareMathOperator{\xor}{\oplus}
  \DeclareMathOperator{\nor}{\downarrow}
  \DeclareMathOperator{\nand}{\uparrow}
  \DeclareMathOperator{\biglor}{\bigvee}
  \DeclareMathOperator{\bigland}{\bigwedge}
  \DeclareMathOperator{\Lr}{\Leftrightarrow}
  \DeclareMathOperator{\lr}{\leftrightarrow}
  \DeclareMathOperator{\ip}{\perp\!\!\!\perp}
  \DeclareMathOperator{\psubset}{\subsetneq}
  \DeclareMathOperator{\psupset}{\supsetneq}
  \DeclareMathOperator{\elsub}{\preceq}
  \DeclareMathOperator{\elsup}{\succeq}
  \DeclareMathOperator{\pelsub}{\prec}
  \DeclareMathOperator{\pelsup}{\succ}
  \DeclareMathOperator{\contr}{\lightning}
  \DeclareMathOperator{\proves}{\vdash}
  \DeclareMathOperator{\nproves}{\nvdash}
  \DeclareMathOperator{\nmodels}{\nvDash}
  \DeclareMathOperator{\forces}{\Vdash}
  \DeclareMathOperator{\nforces}{\nVdash}
  \DeclareMathOperator{\adj}{\dashv}
  \DeclareMathOperator{\restr}{\upharpoonright}
  \DeclareMathOperator{\ex}{\underline{ex}}
  \DeclareMathOperator{\st}{\underline{st}}
  \DeclareMathOperator{\sv}{\underline{sv}}
  \DeclareMathOperator{\tl}{\underline{tl}}
  \DeclareMathOperator{\tensor}{\otimes}
  \DeclareMathOperator{\monus}{\dotdiv}
  \DeclareMathOperator{\Null}{\textsf{NULL}}
  \DeclareMathOperator{\nat}{Nat}
  \DeclareMathOperator{\col}{Col}
  \DeclareMathOperator{\Root}{root}
  \DeclareMathOperator{\aut}{Aut}
  \DeclareMathOperator{\spec}{spec}
  \DeclareMathOperator{\sq}{Sq}
  \DeclareMathOperator{\ann}{ann}
  \DeclareMathOperator{\ffrac}{Frac}
  \DeclareMathOperator{\hull}{Hull}
  \DeclareMathOperator{\chull}{cHull}
  \DeclareMathOperator{\lh}{lh}
  \DeclareMathOperator{\Def}{Def}
  \DeclareMathOperator{\Span}{span}
  \DeclareMathOperator{\form}{Form}
  \DeclareMathOperator{\tc}{trcl}


  % Redeclared operators
  \renewcommand{\P}{\mathcal P}
  \renewcommand{\S}{\mathcal S}
	\renewcommand{\succ}{\text{succ}}
	\renewcommand{\pr}{\text{Pr}}
  \renewcommand{\subset}{\subseteq}
  \renewcommand{\supset}{\supseteq}
  \renewcommand{\nsubset}{\nsubseteq}
  \renewcommand{\nsupset}{\nsupseteq}
  \renewcommand{\hom}{\text{Hom}}
	\renewcommand{\index}{\text{index }}
  \renewcommand{\a}{\b{a}}
	\renewcommand{\r}{{^\omega\omega}}
	\renewcommand{\l}{|}

  % Convenient shortcuts
	\newcommand{\E}{\vec E}
	\newcommand{\bSigma}{\utilde{\b\Sigma}}
	\newcommand{\bPi}{\utilde{\b\Pi}}
	\newcommand{\p}{\mathscr P}
	\newcommand{\pistol}{\mathparagraph}
  \newcommand{\pmax}{\mathbb{P}_{\text{max}}}
	\newcommand{\mw}{\mathcal{M}_{\textsf{mw}}^\sharp}
  \newcommand{\vto}[2]{\begin{pmatrix}#1\\#2\end{pmatrix}}
  \newcommand{\game}[8]{\eq{\begin{array}{ccccccccc} \text{I} & #1 && #3 && #5 && #7\\ \text{II} && #2 && #4 && #6 && #8 \end{array}}}
  \newcommand{\vtre}[3]{\begin{pmatrix}#1\\#2\\#3\end{pmatrix}}
  \newcommand{\mto}[4]{\begin{pmatrix} #1 & #2 \\ #3 & #4\end{pmatrix}}
  \newcommand{\mtre}[9]{\begin{pmatrix} #1 & #2 & #3 \\ #4 & #5 & #6 \\ #7 & #8 & #9\end{pmatrix}}
  \newcommand{\bra}[1]{\langle #1\rangle}
  \newcommand{\dbra}[1]{\llbracket #1 \rrbracket}
  \newcommand{\norm}[1]{\left|\left|#1\right|\right|}
  \newcommand{\abs}[1]{\left|#1\right|}
  \newcommand{\later}{{\vartriangleright}}
  \newcommand{\normal}{\unlhd}
  \newcommand{\ideal}{\unlhd}
  \newcommand{\rel}{\ \text{rel}\ }
  \newcommand{\pnormal}{\mathrel{\ooalign{$\lneq$\cr\raise.22ex\hbox{$\lhd$}\cr}}}
  \newcommand{\pideal}{\mathrel{\ooalign{$\lneq$\cr\raise.22ex\hbox{$\lhd$}\cr}}}
  \newcommand{\acts}{\curvearrowright}
  \newcommand{\colimm}{\varinjlim}
  \newcommand{\limm}{\varprojlim}
  \newcommand{\eff}{\mathcal{E}\! f\! f}
  \newcommand{\set}{\textsf{Set}}
  \newcommand{\fin}{\textsf{Fin }}
  \newcommand{\Top}{\textsf{Top}}
  \newcommand{\zf}{\textsf{ZF}}
  \newcommand{\zfc}{\textsf{ZFC}}
  \newcommand{\ch}{\textsf{CH}}
  \newcommand{\gch}{\textsf{GCH}}
  \newcommand{\ad}{\textsf{AD}}
  \newcommand{\ac}{\textsf{AC}}
  \newcommand{\dc}{\textsf{DC}}
  \newcommand{\cat}{\textsf{Cat}}
  \newcommand{\grp}{\textsf{Grp}}
  \newcommand{\shc}{\mathcal{SHC}}
  \newcommand{\sset}{\textsf{sSet}}
  \newcommand{\gset}{$G$\textsf{Set}}
  \newcommand{\ab}{\textsf{Ab}}
  \newcommand{\godel}[1]{\ulcorner #1 \urcorner}
  \newcommand{\circled}[1]{\raisebox{-0.5pt}{\textcircled{\raisebox{-0.5pt} {{\scriptsize #1}}}}}
  \newcommand{\rmod}{{_R\textsf{Mod}}}
  \newcommand{\modr}{{\textsf{Mod}_R}}
  \newcommand{\lex}{<_{\text{lex}}}
  \newcommand{\po}{\ar@{}[dr]|{\text{\pigpenfont R}}}
  \newcommand{\pb}{\ar@{}[dr]|{\text{\pigpenfont J}}}
  \newcommand{\init}{\unlhd}	
  \newcommand{\pinit}{\lhd}	

\input{/home/leidem/Dropbox/art.tex}

\section{Basic descriptive set theory}

Before we move to infinite games, we introduce some notions from descriptive set theory. This field of set theory studies \textit{definable} subsets of reals.\\

\subsection{The Baire space}
An easier way to work with reals is \textit{not} to work with $\mathbb R$, but to work with the so-called \textbf{Baire space} $\omega^\omega$ of functions $\omega\to\omega$. A convenient way to view this space is as a \textit{tree} with $\omega$ many branches at each node and height $\omega$. The topology on $\omega^\omega$ is generated by the \textbf{basic open sets} $N_s:=\{x\in\omega^\omega\mid s\subset x\}$, where $s\in\omega^{<\omega}:=\bigcup_{n<\omega}\omega^n$.

\qquad The reason why it's okay to work in the Baire space instead of $\mathbb R$ is justified by the fact that the Baire space is homeomorphic to the irrationals $\mathbb R-\mathbb Q$. Since the rationals is a Lebesgue null-set, we're fine with working in these spaces.\\


\subsection{Borel and projective hierachies}

Definability in the context of descriptive set theory is defined via a mixture of topology and measure theory. We define the \textbf{Borel hierachy} as follows, where $\alpha<\omega_1$. A set $A\subset\omega^\omega$ is..
\begin{itemize}
\item $\b\Sigma^0_1$ if it's open in the standard topology;
\item $\b\Pi^0_\alpha$ if $\omega^\omega-A$ is $\b\Sigma^0_\alpha$;
\item $\b\Sigma^0_\alpha$ if there's an $\omega$-sequence of sets $A_0,A_1,A_2,\hdots$ such that $A_i$ is $\b\Pi^0_{\alpha_i}$ for $\alpha_i<\alpha$ and $A=\bigcup_{i<\omega}A_i$;
\item $\b\Delta^0_\alpha$ if it's both $\b\Sigma^0_\alpha$ and $\b\Pi^0_\alpha$.\\
\end{itemize}

Then the \textbf{Borel sets} is the sets $\mathbb B=\bigcup_{\alpha<\omega_1}\b\Sigma^0_\alpha=\bigcup_{\alpha<\omega_1}\b\Pi^0_\alpha$. Alternatively, one could equivalently define $\mathbb B$ as the $\sigma$-algebra generated by the open sets of reals. We can extend this hierachy to the \textbf{projective hierachy}, defined as follows, where $n<\omega$. A set $A\subset\omega^\omega$ is..
\begin{itemize}
\item $\b\Sigma^1_1$ if there exists a surjective continuous function $f:\omega^\omega\to A$;
\item $\b\Pi^1_n$ if $\omega^\omega-A$ is $\b\Sigma^1_n$;
\item $\b\Sigma^1_{n+1}$ if there is a $\b\Pi^1_n$ set $B\subset A\times\omega^\omega$ such that $A$ is the first projection of $B$;
\item $\b\Delta^1_n$ if it's both $\b\Sigma^1_n$ and $\b\Pi^1_n$.\\
\end{itemize}

It's a deep theorem that $\mathbb B=\b\Delta^1_1$, so the projective hierachy is really a continuation of the Borel hierachy.

\exam{\ 
\begin{enumerate}
\item $\b\Pi^0_1$ is exactly the closed sets;
\item $\mathbb Q$ is $\b\Sigma^0_2$;
\item $\mathbb R-\mathbb Q$ is $\b\Pi^0_2$;
\item To construct a $\b\Sigma^0_\alpha$ set for $\alpha> 4$ one \textbf{has} to use axiom of choice, as whether or not every set is $\b\Sigma^0_4$ is independent of $\zf$.
\end{enumerate}
}

\pagebreak
\section{Infinite game theory}

\subsection{Basic game theory}

Given a subset $A\subset\omega^\omega$ we can associate a \textbf{game} $G(A)$, defined as follows. We imagine two players I and II each playing natural numbers $n\in\omega$:
\game{x_0}{x_1}{x_2}{x_3}{x_4}{x_5}{\cdots}{\cdots}

Then the sequence $x:=(x_i)_{i<\omega}$ is an element of $\omega^\omega$ and we say that player I \textbf{wins} if $x\in A$ -- otherwise player II wins. A \textbf{strategy} for player I in $G(A)$ is a function $\sigma:\bigcup_{n<\omega}\omega^{2n}\to\omega$ and a strategy for player II is a function $\tau:\bigcup_{n<\omega}\omega^{2n+1}\to\omega$. We think of such strategies as supplying the given player with his moves. Say, if $\sigma$ is a strategy for player I, then a generic play goes like
\game{\sigma(\bra{})}{y_0}{\sigma(\bra{\sigma(\bra{}),y_0})}{y_1}{\sigma(\bra{\sigma(\bra{\sigma(\bra{}),y_1}),y_1}}{y_2}{\cdots}{\cdots}

We denote this play as $\sigma*y\in\omega^\omega$. A strategy $\sigma$ is \textbf{winning} in $G(A)$ if $\sigma*y\in A$ for every $y\in\omega^\omega$. That is, no matter what player II plays, player I will always win. Strategies for player II are defined analogously. A game $G(A)$ is \textbf{determined} if one of the players has a winning strategy.

\qquad We say that a game $G(A)$ is $\b\Gamma$ if $A$ is $\b\Gamma$, where $\b\Gamma$ is a pointclass -- this could for instance be open sets $\b\Sigma^0_1$, Borel sets $\b\Delta^1_1$ or \textit{analytic sets} $\b\Sigma^1_1$.\\


\subsection{Determinacy}

Why do we care whether or not games are determined? Let's introduce an axiom, called the \textbf{axiom of determinacy}, also written AD, saying that $G(A)$ is determined for every $A\subset\omega^\omega$. It turns out that AD implies that reals are well-behaved:

\qtheo{
\label{LebesgueTheo}
Assuming AD, every set of reals is Lebesgue measurable and satisfies the Continuum Hypothesis (i.e. that for every $A\subset\mathbb R$, either $A$ is countable or $|A|=|\mathbb R|$).
}

This is just a few of the consequences of AD. But AD turns out to be a bit \textit{too} nice: it's inconsistent with AC.

\prop[ZF]{
Assuming AC, there exists a non-determined set $A\subset\mathbb R$.
}
\proof{(sketch)
We'll construct a set which isn't Lebesgue measurable, so that Theorem \ref{LebesgueTheo} implies that AD is false. We'll show the construction and leave out the argument showing that it isn't measurable.

\qquad Consider the quotient group $\mathbb R/\mathbb Q$. Use the axiom of choice to ensure that a subset $X\subset[0,1]$ exists with exactly one element from each equivalence class $[x]\in\mathbb R/\mathbb Q$. Such a set is called a \textit{Vitali set}.
}

But what if we restrict ourselves to a smaller class of sets? Could we find such a class in which all sets in the class are determined, but it doesn't contradict AC? Let's start with the simplest sets, according to our hierarchies: the open sets.

\theo[Gale-Stewart '53]{
Every open game $G(A)$ is determined.
}
\proof{
Assume that player I has no winning strategy in $G(A)$. Say that a partial play $\bra{x_0,\hdots,x_{2n}}\in\omega^{<\omega}$ is \textbf{not losing for player II} if player I has no winning strategy from that point on. Note that by definition $\bra{}$ is not losing for player II.

\qquad Assume now that player II is not losing at $p:=\bra{x_0,\hdots,x_{2n}}$. We claim that there exists some $x_{2n+1}$ such that for every $x_{2n+2}$, $p\cc\bra{x_{2n+1},x_{2n+2}}$ is not losing for player II. Indeed, assuming it wasn't the case we would have that no matter what player II played, player I would have a play such that player I would have a winning strategy at that point. But this means exactly that player I had a winning strategy at $p$, so $p$ would be losing for player II, $\contr$. Define the strategy $\tau$ for player II as these ``non-losing'' moves.

\qquad We now claim that $\tau$ is a winning strategy for player II. Fix some $x\in\omega^\omega$. Then $(x*\tau)\restr{2k+1}$ is not losing for player II for every $k<\omega$. Assume for a contradiction that $x*\tau\in A$. Then since $A$ is open, find an open neighbourhood $N_q\subset A$ of $x*\tau$ satisfying that $\len(q)$ is odd. But now $q=(x*\tau)\restr{2k+1}$ for some $k<\omega$, so $q$ is both losing and not losing for player II, $\contr$. Hence $x*\tau\notin A$ and $\tau$ is winning for player II.
}

Over 20 years later, Martin improved this result greatly, to \textit{all} Borel sets.

\qtheo[Martin '75]{
Every Borel game $G(A)$ is determined.
}

The proof of this subliminal result was almost 10 pages long, and improved upon in '82 to a purely inductive argument, shortening it to about 5 pages. The proof features an ingenious idea of providing a general sufficient condition for games to be determined, called \textit{unraveling}. Details can be read in my project.

\qquad What about the next step, analytic sets? This turns out to be independent of $\zfc$. But under strong assumptions, it turns out that we can prove it anyway. Recall from my last talk that a \textbf{measurable cardinal} exists iff there exists an elementary embedding $j:V\to M$ for some universe $M$ in which $\zfc$ holds.

\qtheo[Martin '70]{
If there exists a measurable cardinal then every analytic game $G(A)$ is determined.
}

So.. what about the rest of the projective hierarchy? This turns out to be unprovable even assuming a measurable. However, there is a strictly stronger notion of a measurable cardinal called a \textbf{Woodin cardinal}\footnote{The precise definition is as follows. $\kappa$ is Woodin if for every function $f:\kappa\to\kappa$ there's a cardinal $\lambda<\kappa$, a transitive inner model $M$ and an elementary embedding $j:V\to M$ such that $f$ restricts to a function $f\restr\lambda:\lambda\to\lambda$, $\crit(j)=\lambda$ and $V_{j(f)(\lambda)}\subset M$.}, which helps us.

\qtheo[Martin, Steel '85]{
If there exists infinitely many Woodin cardinals, then every projective game $G(A)$ is determined.
}

Okay, so we can subsume our entire hierarchy assuming these Woodin cardinals exist. But, can we do more? Recall the definition of a constructible set from my last talk, which is a set that can be described by a formula. Every projective set can (by definition) be described by a formula. Is every constructible set determined?

\qtheo[Woodin '85]{
If there exists infinitely many Woodin cardinals with a measurable above them, then every constructible game $G(A)$ is determined.
}

\end{document}
