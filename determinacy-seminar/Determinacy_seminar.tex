\newcommand{\mytitle}{Determinacy -- seminar}
\input{/home/leidem/Dropbox/std_preamble.tex}
\input{/home/leidem/Dropbox/art.tex}

\section{Basic descriptive set theory}

Before we move to infinite games, we introduce some notions from descriptive set theory. This field of set theory studies \textit{definable} subsets of reals.\\

\subsection{The Baire space}
An easier way to work with reals is \textit{not} to work with $\mathbb R$, but to work with the so-called \textbf{Baire space} $\omega^\omega$ of functions $\omega\to\omega$. A convenient way to view this space is as a \textit{tree} with $\omega$ many branches at each node and height $\omega$. The topology on $\omega^\omega$ is generated by the \textbf{basic open sets} $N_s:=\{x\in\omega^\omega\mid s\subset x\}$, where $s\in\omega^{<\omega}:=\bigcup_{n<\omega}\omega^n$.

\qquad The reason why it's okay to work in the Baire space instead of $\mathbb R$ is justified by the fact that the Baire space is homeomorphic to the irrationals $\mathbb R-\mathbb Q$. Since the rationals is a Lebesgue null-set, we're fine with working in these spaces.\\


\subsection{Borel and projective hierachies}

Definability in the context of descriptive set theory is defined via a mixture of topology and measure theory. We define the \textbf{Borel hierachy} as follows, where $\alpha<\omega_1$. A set $A\subset\omega^\omega$ is..
\begin{itemize}
\item $\b\Sigma^0_1$ if it's open in the standard topology;
\item $\b\Pi^0_\alpha$ if $\omega^\omega-A$ is $\b\Sigma^0_\alpha$;
\item $\b\Sigma^0_\alpha$ if there's an $\omega$-sequence of sets $A_0,A_1,A_2,\hdots$ such that $A_i$ is $\b\Pi^0_{\alpha_i}$ for $\alpha_i<\alpha$ and $A=\bigcup_{i<\omega}A_i$;
\item $\b\Delta^0_\alpha$ if it's both $\b\Sigma^0_\alpha$ and $\b\Pi^0_\alpha$.\\
\end{itemize}

Then the \textbf{Borel sets} is the sets $\mathbb B=\bigcup_{\alpha<\omega_1}\b\Sigma^0_\alpha=\bigcup_{\alpha<\omega_1}\b\Pi^0_\alpha$. Alternatively, one could equivalently define $\mathbb B$ as the $\sigma$-algebra generated by the open sets of reals. We can extend this hierachy to the \textbf{projective hierachy}, defined as follows, where $n<\omega$. A set $A\subset\omega^\omega$ is..
\begin{itemize}
\item $\b\Sigma^1_1$ if there exists a surjective continuous function $f:\omega^\omega\to A$;
\item $\b\Pi^1_n$ if $\omega^\omega-A$ is $\b\Sigma^1_n$;
\item $\b\Sigma^1_{n+1}$ if there is a $\b\Pi^1_n$ set $B\subset A\times\omega^\omega$ such that $A$ is the first projection of $B$;
\item $\b\Delta^1_n$ if it's both $\b\Sigma^1_n$ and $\b\Pi^1_n$.\\
\end{itemize}

It's a deep theorem that $\mathbb B=\b\Delta^1_1$, so the projective hierachy is really a continuation of the Borel hierachy.

\exam{\ 
\begin{enumerate}
\item $\b\Pi^0_1$ is exactly the closed sets;
\item $\mathbb Q$ is $\b\Sigma^0_2$;
\item $\mathbb R-\mathbb Q$ is $\b\Pi^0_2$;
\item To construct a $\b\Sigma^0_\alpha$ set for $\alpha> 4$ one \textbf{has} to use axiom of choice, as whether or not every set is $\b\Sigma^0_4$ is independent of $\zf$.
\end{enumerate}
}

\pagebreak
\section{Infinite game theory}

\subsection{Basic game theory}

Given a subset $A\subset\omega^\omega$ we can associate a \textbf{game} $G(A)$, defined as follows. We imagine two players I and II each playing natural numbers $n\in\omega$:
\game{x_0}{x_1}{x_2}{x_3}{x_4}{x_5}{\cdots}{\cdots}

Then the sequence $x:=(x_i)_{i<\omega}$ is an element of $\omega^\omega$ and we say that player I \textbf{wins} if $x\in A$ -- otherwise player II wins. A \textbf{strategy} for player I in $G(A)$ is a function $\sigma:\bigcup_{n<\omega}\omega^{2n}\to\omega$ and a strategy for player II is a function $\tau:\bigcup_{n<\omega}\omega^{2n+1}\to\omega$. We think of such strategies as supplying the given player with his moves. Say, if $\sigma$ is a strategy for player I, then a generic play goes like
\game{\sigma(\bra{})}{y_0}{\sigma(\bra{\sigma(\bra{}),y_0})}{y_1}{\sigma(\bra{\sigma(\bra{\sigma(\bra{}),y_1}),y_1}}{y_2}{\cdots}{\cdots}

We denote this play as $\sigma*y\in\omega^\omega$. A strategy $\sigma$ is \textbf{winning} in $G(A)$ if $\sigma*y\in A$ for every $y\in\omega^\omega$. That is, no matter what player II plays, player I will always win. Strategies for player II are defined analogously. A game $G(A)$ is \textbf{determined} if one of the players has a winning strategy.

\qquad We say that a game $G(A)$ is $\b\Gamma$ if $A$ is $\b\Gamma$, where $\b\Gamma$ is a pointclass -- this could for instance be open sets $\b\Sigma^0_1$, Borel sets $\b\Delta^1_1$ or \textit{analytic sets} $\b\Sigma^1_1$.\\


\subsection{Determinacy}

Why do we care whether or not games are determined? Let's introduce an axiom, called the \textbf{axiom of determinacy}, also written AD, saying that $G(A)$ is determined for every $A\subset\omega^\omega$. It turns out that AD implies that reals are well-behaved:

\qtheo{
\label{LebesgueTheo}
Assuming AD, every set of reals is Lebesgue measurable and satisfies the Continuum Hypothesis (i.e. that for every $A\subset\mathbb R$, either $A$ is countable or $|A|=|\mathbb R|$).
}

This is just a few of the consequences of AD. But AD turns out to be a bit \textit{too} nice: it's inconsistent with AC.

\prop[ZF]{
Assuming AC, there exists a non-determined set $A\subset\mathbb R$.
}
\proof{(sketch)
We'll construct a set which isn't Lebesgue measurable, so that Theorem \ref{LebesgueTheo} implies that AD is false. We'll show the construction and leave out the argument showing that it isn't measurable.

\qquad Consider the quotient group $\mathbb R/\mathbb Q$. Use the axiom of choice to ensure that a subset $X\subset[0,1]$ exists with exactly one element from each equivalence class $[x]\in\mathbb R/\mathbb Q$. Such a set is called a \textit{Vitali set}.
}

But what if we restrict ourselves to a smaller class of sets? Could we find such a class in which all sets in the class are determined, but it doesn't contradict AC? Let's start with the simplest sets, according to our hierarchies: the open sets.

\theo[Gale-Stewart '53]{
Every open game $G(A)$ is determined.
}
\proof{
Assume that player I has no winning strategy in $G(A)$. Say that a partial play $\bra{x_0,\hdots,x_{2n}}\in\omega^{<\omega}$ is \textbf{not losing for player II} if player I has no winning strategy from that point on. Note that by definition $\bra{}$ is not losing for player II.

\qquad Assume now that player II is not losing at $p:=\bra{x_0,\hdots,x_{2n}}$. We claim that there exists some $x_{2n+1}$ such that for every $x_{2n+2}$, $p\cc\bra{x_{2n+1},x_{2n+2}}$ is not losing for player II. Indeed, assuming it wasn't the case we would have that no matter what player II played, player I would have a play such that player I would have a winning strategy at that point. But this means exactly that player I had a winning strategy at $p$, so $p$ would be losing for player II, $\contr$. Define the strategy $\tau$ for player II as these ``non-losing'' moves.

\qquad We now claim that $\tau$ is a winning strategy for player II. Fix some $x\in\omega^\omega$. Then $(x*\tau)\restr{2k+1}$ is not losing for player II for every $k<\omega$. Assume for a contradiction that $x*\tau\in A$. Then since $A$ is open, find an open neighbourhood $N_q\subset A$ of $x*\tau$ satisfying that $\len(q)$ is odd. But now $q=(x*\tau)\restr{2k+1}$ for some $k<\omega$, so $q$ is both losing and not losing for player II, $\contr$. Hence $x*\tau\notin A$ and $\tau$ is winning for player II.
}

Over 20 years later, Martin improved this result greatly, to \textit{all} Borel sets.

\qtheo[Martin '75]{
Every Borel game $G(A)$ is determined.
}

The proof of this subliminal result was almost 10 pages long, and improved upon in '82 to a purely inductive argument, shortening it to about 5 pages. The proof features an ingenious idea of providing a general sufficient condition for games to be determined, called \textit{unraveling}. Details can be read in my project.

\qquad What about the next step, analytic sets? This turns out to be independent of $\zfc$. But under strong assumptions, it turns out that we can prove it anyway. Recall from my last talk that a \textbf{measurable cardinal} exists iff there exists an elementary embedding $j:V\to M$ for some universe $M$ in which $\zfc$ holds.

\qtheo[Martin '70]{
If there exists a measurable cardinal then every analytic game $G(A)$ is determined.
}

So.. what about the rest of the projective hierarchy? This turns out to be unprovable even assuming a measurable. However, there is a strictly stronger notion of a measurable cardinal called a \textbf{Woodin cardinal}\footnote{The precise definition is as follows. $\kappa$ is Woodin if for every function $f:\kappa\to\kappa$ there's a cardinal $\lambda<\kappa$, a transitive inner model $M$ and an elementary embedding $j:V\to M$ such that $f$ restricts to a function $f\restr\lambda:\lambda\to\lambda$, $\crit(j)=\lambda$ and $V_{j(f)(\lambda)}\subset M$.}, which helps us.

\qtheo[Martin, Steel '85]{
If there exists infinitely many Woodin cardinals, then every projective game $G(A)$ is determined.
}

Okay, so we can subsume our entire hierarchy assuming these Woodin cardinals exist. But, can we do more? Recall the definition of a constructible set from my last talk, which is a set that can be described by a formula. Every projective set can (by definition) be described by a formula. Is every constructible set determined?

\qtheo[Woodin '85]{
If there exists infinitely many Woodin cardinals with a measurable above them, then every constructible game $G(A)$ is determined.
}

\end{document}
