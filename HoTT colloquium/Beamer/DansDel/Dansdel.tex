\documentclass{beamer}
\usetheme{Warsaw}
\setbeamercovered{transparent}
\usepackage[utf8x]{inputenc}
\usepackage{proof}
\usepackage{tikz} 
%\usepackage{tikz-cd}
\usetikzlibrary{matrix,arrows}
\newcommand{\eq}[1]{\begin{align*} #1 \end{align*}}
\newcommand{\bra}[1]{\langle #1\rangle}
\newcommand{\fst}{\mathsf{fst }}
\newcommand{\snd}{\mathsf{snd }}
\newcommand{\NN}{\mathbb{N}}
\newcommand{\SSS}{\mathbb{S}}
\DeclareMathOperator{\dom}{dom}	
\beamertemplatetransparentcoveredhigh

\title[Homotopy Type Theory]{
	An introduction to\\
	Homotopy Type Theory\\
}
\author{Dan Saattrup Nielsen \& Martin Speirs}
\date{}

\begin{document}

\begin{frame}
	\titlepage
\end{frame}

% TOC
\begin{frame}{What's going to happen?}
	\begin{itemize}
	\pause\item What is homotopy type theory?
	\pause\item Basic type theory
	\pause\item A proof of the axiom of choice
	\pause\item Isomorphic objects are equal
	\pause\item Higher inductive types
	\pause\item Fundamental group of the circle is the integers
	\end{itemize}
\end{frame}

\begin{frame}{What is homotopy type theory?}
	\begin{itemize}
	\item It's a \textit{synthetic} approach to homotopy theory
	\pause\item It's a \textit{foundational} theory
	\pause\item Connections with category theory, topology and logic
	\pause\item Propositions as types
	\pause\item HoTT = ITT + UA + HITs
	\end{itemize}
\end{frame}

\begin{frame}{Type theory}
	\begin{itemize}
	\pause\item Type theory is \textit{syntactical}
	\pause\item Type theory is an \textit{independent} foundation - no logical foundation is needed
	\pause\item Type theory structure:
	\begin{itemize}
		\pause\item Types $A$
		\pause\item Terms $a:A$
		\pause\item Dependent types $A(x)$
		\pause\item Dependent terms $a(x):A(x)$
		\pause\item $A\times B$, $A\to B$, $\Sigma_{(a:A)}B(a)$, $\Pi_{(a:A)}B(a)$, $\hdots$
	\end{itemize}
	\pause\item Type theory builds on \textit{rules} rather than \textit{axioms}
	\end{itemize}
\end{frame}

\begin{frame}{Axiom of choice}
	\begin{itemize}
	\pause\item Regular AC: ``Given a family of non-empty sets $F$, there exists a choice function $g$ with domain $F$, satisfying $g(X)\in X$ for all $X\in F$.''
	\pause\item In regular logic:\\
		$(\forall X\in F)(\exists x\in X)\Rightarrow$\\
		$(\exists g)(``g\text{ function}"\land\dom g=F\land(\forall X\in F)(g(X)\in X))$
	\end{itemize}
	\pause\begin{block}{Theorem of Choice}
	$\left(\prod_{(X:F)}\sum_{(x:X)}P(x,X)\right)\to\pause\left(\sum_{(g:\prod_{(X:F)}X)}\prod_{(X:F)}P(g(X),X)\right)$
	\end{block}
	\pause\begin{block}{Proof}
	Take $\lambda f.\pause\bra{\lambda X.\pause\fst f(X),\pause\lambda X.\pause\snd f(X)}$.\pause\qed
	\end{block}
\end{frame}

\begin{frame}{The Identity type}
	\begin{itemize}
	\pause\item The identity type $Id_A(x,y)$ (also written $x=_Ay$) for $x,y : A$ %, and even $x=y$) for $x,y:A$
	\pause\item $p:Id_A(x,y)$ is a \textit{proof} of $x=y$, or a \textit{path} from $x$ to $y$	
	\pause\item Special element, $refl_x : Id_A(x,x)$ (``constant path at $x$'')
	\pause\item What is $Id_{Id_A}(p,q)$?
	\end{itemize}
\end{frame}

\begin{frame}{Univalence}
	\begin{itemize}
	\pause\item Isomorphism type $X\cong Y$ for $X,Y$ types:\\
		$\sum_{(f:X\to Y)}\sum_{(g:Y\to X)}$\\
		$\left(\prod_{(x:X)}g(f(x))=x\right)\times\left(\prod_{(y:Y)}f(g(y))=y\right)$
	\pause\item $p:X\cong Y$ is a \textit{proof} of $X\cong Y$
	\end{itemize}
	\pause\begin{block}{Univalence axiom}
		For all types $X,Y$ it holds that $(X=Y)\cong(X\cong Y)$; i.e. ``isomorphic objects are equal''.
	\end{block}
\end{frame}

\end{document}