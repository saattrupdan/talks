% Document definition
\documentclass[a4paper,10pt]{article}
% Packages
% General packages
  \usepackage[T1]{fontenc}
  \usepackage[latin1]{inputenc} % Danish language support
  \usepackage{sectsty} % Makes it possible to manipulate fonts
  \usepackage[light]{antpolt} % Provides awesome font
  \usepackage{hyperref} % Provides \url and clickable links in the pdf
  %\usepackage{cleveref}	% Provides \cref and \Cref for including text before references
  \usepackage{graphicx} % Provides the \includegraphics[]{} command
  \usepackage{comment} % Provides comment-environment for multi-line commenting
  \usepackage{twoopt} % Allows adding commands with two optional arguments
  \usepackage{setspace} % Provides onehalfspacing environment
  \usepackage{enumitem} % Provides control of spacing in lists
  \usepackage{amsbsy} % Provides \boldsymbol command
  \usepackage{apacite} % Provides bibliography style
  \usepackage[sectionbib]{natbib} % Provides \citep and \citeyearpar commands
  \usepackage{todonotes} % Provides to do command
  \usepackage{wrapfig} % For wrapping figures
  %\usepackage{faktor} % Allows for quotients like A/B **PRODUCES ERROR ON DAN'S OFFICE COMPUTER**
  \usepackage{makeidx} % Enables index
  \usepackage{tocloft} % Allows changing spacing in TOC
	
  % Math packages
  \usepackage{amsmath, amssymb, amsfonts} % Math symbolic jargon
  \usepackage{amsthm} % Theorem environment
  \usepackage{mathrsfs}	% Provides the \mathscr{} curly font
  \usepackage{stmaryrd}	% Provides the \lightning symbol and semantics-brackets, among others
  \usepackage{mathabx} % Provides the \dotdiv symbol	
  \usepackage{proof} % Natural deduction with \infer
  \usepackage{tikz} % Awesome diagrams
    \usetikzlibrary{cd} % Commutative diagrams 
    \usetikzlibrary{matrix,arrows} % Matrices and arrows for style points
    \usetikzlibrary{decorations.pathreplacing,calc,arrows.meta} % For making tree cds
  \usepackage[all]{xy} % Provides xymatrix environment for diagrams
		\newdir{|>}{-<0pt,5pt>{\blacktriangleup}}
  \usepackage{pigpen} % Provides pullback/pushout symbols in diagrams
  \usepackage{xfrac} % Provides \sfrac command for diagonal fraction
	%\usepackage{undertilde} % Provides \utilde command
		
  % Page layout
  %\usepackage{fullpage} % Reduces margins
  \usepackage{wallpaper} % Adds the \ThisLRCornerWallPaper command
  \usepackage{fancyhdr}	% Provides headers
  %\usepackage[compact,small]{titlesec} % Makes titles a bit smaller with shorter proceeding space
  \usepackage{changepage} % Provides adjustwidth environment for claims

  %\usepackage{xargs} % better handling of function/command arguments **PRODUCES ERROR ON DAN'S OFFICE COMPUTER**

  % Todonotes
  \usepackage{todonotes} % Enables the \todo command
  \newcommand{\todocomment}[2][]{\todo[linecolor=blue,backgroundcolor=blue!25,bordercolor=blue,#1]{#2}}

% Tree diagrams
\tikzset{
  tree/.style 2 args={
    decorate,
    decoration={
      show path construction,
      lineto code={
        \draw[dotted,-] (\tikzinputsegmentfirst) --($(\tikzinputsegmentfirst)!.5!(\tikzinputsegmentlast)$);
        \draw[-{Latex}] ($(\tikzinputsegmentfirst)!.5!(\tikzinputsegmentlast)$) --(\tikzinputsegmentlast) node [midway,right] {\small{$#1$}};
        \draw[-] (\tikzinputsegmentfirst) --++ (105:0.65cm);
        \draw[-] (\tikzinputsegmentfirst) --++ (75:0.65cm) node [midway, right] {\small{$#2$}};
      }
    }
  }
}

\tikzset{
  treeplain/.style 2 args={
    decorate,
    decoration={
      show path construction,
      lineto code={
        \draw[dotted,-] (\tikzinputsegmentfirst) --($(\tikzinputsegmentfirst)!.5!(\tikzinputsegmentlast)$);
        \draw[-] ($(\tikzinputsegmentfirst)!.5!(\tikzinputsegmentlast)$) --(\tikzinputsegmentlast) node [midway,right] {\small{$#1$}};
        \draw[-] (\tikzinputsegmentfirst) --++ (105:0.65cm);
        \draw[-] (\tikzinputsegmentfirst) --++ (75:0.65cm) node [midway, right] {\small{$#2$}};
      }
    }
  }
}

% Make \emph boldface
\let\emph\relax % there's no \RedeclareTextFontCommand
\DeclareTextFontCommand{\emph}{\bfseries}

% Prevent widows and orphans
\widowpenalty = 10000
\clubpenalty = 10000

% List spacing
\setlist{nolistsep}

% No indent
\setlength{\parindent}{0ex}

% Line spacing (1.3 = one and a half spacing)
\linespread{1.3}

% Theorem environment
\newtheoremstyle{scthmstyle} % Name
	{15pt} % Space above
	{15pt} % Space below
	{\itshape} % Body font
	{} % Indent amount
	{\bfseries\scshape} % Theorem head font
	{.} % Punctuation after theorem head
	{.5em} % Space after theorem head
	{} % Theorem head spec (can be left empty, meaning ënormalí)

\newtheoremstyle{scdefstyle} % Name
	{15pt} % Space above
	{15pt} % Space below
	{\normalfont} % Body font
	{} % Indent amount
	{\bfseries\scshape} % Theorem head font
	{.} % Punctuation after theorem head
	{.5em} % Space after theorem head
	{} % Theorem head spec (can be left empty, meaning ënormalí)

\newtheoremstyle{scremstyle} % Name
	{15pt} % Space above
	{15pt} % Space below
	{\normalfont} % Body font
	{} % Indent amount
	{\itshape} % Theorem head font
	{.} % Punctuation after theorem head
	{.5em} % Space after theorem head
	{} % Theorem head spec (can be left empty, meaning ënormalí)

\newtheoremstyle{scclaistyle} % Name
	{15pt} % Space above
	{15pt} % Space below
	{\normalfont} % Body font
	{0.5cm} % Indent amount
	{\itshape} % Theorem head font
	{.} % Punctuation after theorem head
	{.5em} % Space after theorem head
	{} % Theorem head spec (can be left empty, meaning ënormalí)

\theoremstyle{scthmstyle}
\newtheorem{theorem}{Theorem}[section]
\newtheorem{proposition}[theorem]{Proposition}
\newtheorem{lemma}[theorem]{Lemma}
\newtheorem{corollary}[theorem]{Corollary}
\theoremstyle{scdefstyle}
\newtheorem{definition}[theorem]{Definition}
\newtheorem{convention}[theorem]{Convention}
\newtheorem{example}[theorem]{Example}
\newtheorem{question}[theorem]{Question}
\theoremstyle{scremstyle}
\newtheorem{remark}[theorem]{Remark}
\theoremstyle{scclaistyle}
\newtheorem{claim}{Claim}[theorem]

% User-defined commands
  % General things	
  \newcommand{\eq}[1]{\begin{align*} #1 \end{align*}}
  \newcommand{\eqq}[1]{\begin{align*} #1\\ \end{align*}}
  \newcommand{\pix}[2]{\begin{center}\includegraphics[scale=#2]{#1}\\\end{center}}
  \newcommand{\cd}[1]{\eq{\xymatrix{#1}}}
  \renewcommand{\labelenumi}{(\roman{enumi}) } % Using roman numerals in lists
  \renewcommand{\b}[1]{{\bf #1}}
  \renewcommand{\abstract}[1]{\begin{quote}{\footnotesize\textsc{Abstract.} #1}\\\end{quote}}

  % Theorem environments
  \newcommandtwoopt{\theo}[3][][]{
    \begin{theorem}[#1]\label{#2}
      #3
    \end{theorem}}
  \newcommandtwoopt{\prop}[3][][]{
    \begin{proposition}[#1]\label{#2}
      #3
    \end{proposition}}
  \newcommandtwoopt{\lemm}[3][][]{
    \begin{lemma}[#1]\label{#2}
      #3
    \end{lemma}}
  \newcommandtwoopt{\coro}[3][][]{
    \begin{corollary}[#1]\label{#2}
      #3
    \end{corollary}}
  \newcommandtwoopt{\defi}[3][][]{
    \begin{definition}[#1]\label{#2}
      #3$\hfill\dashv$
    \end{definition}}
  \newcommandtwoopt{\exam}[3][][]{
    \begin{example}[#1]\label{#2}
      #3
    \end{example}}
  \newcommandtwoopt{\ques}[3][][]{
    \begin{question}[#1]\label{#2}
      #3
    \end{question}}
  \newcommandtwoopt{\rema}[3][][]{
    \begin{remark}[#1]\label{#2}
      #3
    \end{remark}}

  \newcommandtwoopt{\qtheo}[3][][]{
    \begin{theorem}[#1]\label{#2}
      #3$\hfill\dashv$
    \end{theorem}}
  \newcommandtwoopt{\qprop}[3][][]{
    \begin{proposition}[#1]\label{#2}
      #3$\hfill\dashv$
    \end{proposition}}
  \newcommandtwoopt{\qlemm}[3][][]{
    \begin{lemma}[#1]\label{#2}
      #3$\hfill\dashv$
    \end{lemma}}
  \newcommandtwoopt{\qcoro}[3][][]{
    \begin{corollary}[#1]\label{#2}
      #3$\hfill\dashv$
    \end{corollary}}

  \newcommandtwoopt{\defin}[3][][]{
    \begin{definition}[#1]\label{#2}
      #3
    \end{definition}}

  \newcommand{\proofretard}{\textsc{Proof.} }
  \renewcommand{\proof}[1]{\textsc{Proof.} #1$\qed$\\}
  \newcommand{\clai}[2][]{\begin{claim}[#1]#2\end{claim}}
  \newcommand{\cproof}[1]{
    \begin{adjustwidth}{0.5cm}{0pt}
      \textsc{Proof of claim.} #1$\hfill\dashv$\\
    \end{adjustwidth}}
  \renewcommand{\qed}{\hfill\blacksquare}
  \newcommand{\qedeq}{\tag*{$\blacksquare$}}

  % Declared operators
  \DeclareMathOperator{\dirlim}{dirlim}
  \DeclareMathOperator{\iterates}{I}
  \DeclareMathOperator{\code}{Code}
  \DeclareMathOperator{\piterates}{pI}
  \DeclareMathOperator{\blowups}{B}
  \DeclareMathOperator{\pblowups}{pB}
  \DeclareMathOperator{\lhod}{\triangleleft_{\mathrm{HOD}}}
  \DeclareMathOperator{\lehod}{\trianglelefteq_{\mathrm{HOD}}}
  \DeclareMathOperator{\ledj}{\le_{\mathrm{DJ}}}
  \DeclareMathOperator{\nledj}{\not{\le}_{\mathrm{DJ}}}
  \DeclareMathOperator{\ldj}{<_{\mathrm{DJ}}}
  \DeclareMathOperator{\di}{\textsf{DI}}
  \DeclareMathOperator{\hc}{\textsf{HC}}
  \DeclareMathOperator{\M}{\mathcal M}
  \DeclareMathOperator{\N}{\mathcal N}
  \DeclareMathOperator{\K}{\mathcal K}
  \DeclareMathOperator{\F}{\mathcal F}
  \DeclareMathOperator{\Q}{\mathcal Q}
  \DeclareMathOperator{\W}{\mathcal W}
  \DeclareMathOperator{\V}{\mathcal V}
  \DeclareMathOperator{\T}{\mathcal T}
  \DeclareMathOperator{\U}{\mathcal U}
  \DeclareMathOperator{\B}{\mathcal B}
  \DeclareMathOperator{\R}{\mathcal R}
  \DeclareMathOperator{\D}{\mathcal D}
  \DeclareMathOperator{\G}{\mathcal G}
  \DeclareMathOperator{\C}{\mathcal C}
  \DeclareMathOperator{\A}{\mathcal A}
  \DeclareMathOperator{\h}{\mathcal H}
  \DeclareMathOperator{\core}{\mathfrak C}
  \DeclareMathOperator{\mc}{\textsf{MC}}
  \DeclareMathOperator{\refl}{Refl}
  \DeclareMathOperator{\pp}{pp}
  \DeclareMathOperator{\on}{On}
  \DeclareMathOperator{\rk}{rk}
  \DeclareMathOperator{\otp}{otp}
  \DeclareMathOperator{\pr}{pr}
  \DeclareMathOperator{\lp}{Lp}
  \DeclareMathOperator{\In}{in}
  \DeclareMathOperator{\sgn}{sgn}
  \DeclareMathOperator{\lcm}{lcm}
  \DeclareMathOperator{\ran}{ran}
  \DeclareMathOperator{\cod}{cod}
  \DeclareMathOperator{\dom}{dom}	
  \DeclareMathOperator{\cond}{cond}
  \DeclareMathOperator{\con}{Con}
  \DeclareMathOperator{\rank}{rank}
  \DeclareMathOperator{\gal}{Gal}
  \DeclareMathOperator{\cov}{Cov}
  \DeclareMathOperator{\im}{im}
  \DeclareMathOperator{\sub}{Sub}
  \DeclareMathOperator{\diam}{diam}
  \DeclareMathOperator{\hod}{HOD}
  \DeclareMathOperator{\od}{OD}
  \DeclareMathOperator{\codom}{codom}
  \DeclareMathOperator{\Det}{Det}
  \DeclareMathOperator{\len}{len}
  \DeclareMathOperator{\ma}{MA}
  \DeclareMathOperator{\id}{id}
  \DeclareMathOperator{\sing}{Sing}
  \DeclareMathOperator{\pred}{pred}
  \DeclareMathOperator{\cl}{cl}
  \DeclareMathOperator{\Int}{int}
  \DeclareMathOperator{\ob}{Ob}
  \DeclareMathOperator{\mor}{Mor}
  \DeclareMathOperator{\sh}{Sh}
  \DeclareMathOperator{\ot}{ot}
  \DeclareMathOperator{\ult}{Ult}
  \DeclareMathOperator{\sg}{sg}
  \DeclareMathOperator{\env}{Env}
  \DeclareMathOperator{\tor}{Tor}
  \DeclareMathOperator{\ext}{Ext}
  \DeclareMathOperator{\comp}{\textsf{Comp}}
  \DeclareMathOperator{\card}{card}
  \DeclareMathOperator{\cf}{cf}
  \DeclareMathOperator{\cof}{cof}
  \DeclareMathOperator{\cc}{\text{\textasciicircum}}
  \DeclareMathOperator{\sk}{sk}
  \DeclareMathOperator{\crit}{crit}
  \DeclareMathOperator{\cls}{cls}
  \DeclareMathOperator{\pd}{pd}
  \DeclareMathOperator{\ev}{ev}
  \DeclareMathOperator{\wo}{WO}
  \DeclareMathOperator{\wfp}{wfp}
  \DeclareMathOperator{\xor}{\oplus}
  \DeclareMathOperator{\nor}{\downarrow}
  \DeclareMathOperator{\nand}{\uparrow}
  \DeclareMathOperator{\biglor}{\bigvee}
  \DeclareMathOperator{\bigland}{\bigwedge}
  \DeclareMathOperator{\Lr}{\Leftrightarrow}
  \DeclareMathOperator{\lr}{\leftrightarrow}
  \DeclareMathOperator{\ip}{\perp\!\!\!\perp}
  \DeclareMathOperator{\psubset}{\subsetneq}
  \DeclareMathOperator{\psupset}{\supsetneq}
  \DeclareMathOperator{\elsub}{\preceq}
  \DeclareMathOperator{\elsup}{\succeq}
  \DeclareMathOperator{\pelsub}{\prec}
  \DeclareMathOperator{\pelsup}{\succ}
  \DeclareMathOperator{\contr}{\lightning}
  \DeclareMathOperator{\proves}{\vdash}
  \DeclareMathOperator{\nproves}{\nvdash}
  \DeclareMathOperator{\nmodels}{\nvDash}
  \DeclareMathOperator{\forces}{\Vdash}
  \DeclareMathOperator{\nforces}{\nVdash}
  \DeclareMathOperator{\adj}{\dashv}
  \DeclareMathOperator{\restr}{\upharpoonright}
  \DeclareMathOperator{\ex}{\underline{ex}}
  \DeclareMathOperator{\st}{\underline{st}}
  \DeclareMathOperator{\sv}{\underline{sv}}
  \DeclareMathOperator{\tl}{\underline{tl}}
  \DeclareMathOperator{\tensor}{\otimes}
  \DeclareMathOperator{\monus}{\dotdiv}
  \DeclareMathOperator{\Null}{\textsf{NULL}}
  \DeclareMathOperator{\nat}{Nat}
  \DeclareMathOperator{\col}{Col}
  \DeclareMathOperator{\Root}{root}
  \DeclareMathOperator{\aut}{Aut}
  \DeclareMathOperator{\spec}{spec}
  \DeclareMathOperator{\sq}{Sq}
  \DeclareMathOperator{\ann}{ann}
  \DeclareMathOperator{\ffrac}{Frac}
  \DeclareMathOperator{\hull}{Hull}
  \DeclareMathOperator{\chull}{cHull}
  \DeclareMathOperator{\lh}{lh}
  \DeclareMathOperator{\Def}{Def}
  \DeclareMathOperator{\Span}{span}
  \DeclareMathOperator{\form}{Form}
  \DeclareMathOperator{\tc}{trcl}


  % Redeclared operators
  \renewcommand{\P}{\mathcal P}
  \renewcommand{\S}{\mathcal S}
	\renewcommand{\succ}{\text{succ}}
	\renewcommand{\pr}{\text{Pr}}
  \renewcommand{\subset}{\subseteq}
  \renewcommand{\supset}{\supseteq}
  \renewcommand{\nsubset}{\nsubseteq}
  \renewcommand{\nsupset}{\nsupseteq}
  \renewcommand{\hom}{\text{Hom}}
	\renewcommand{\index}{\text{index }}
  \renewcommand{\a}{\b{a}}
	\renewcommand{\r}{{^\omega\omega}}
	\renewcommand{\l}{|}

  % Convenient shortcuts
	\newcommand{\E}{\vec E}
	\newcommand{\bSigma}{\utilde{\b\Sigma}}
	\newcommand{\bPi}{\utilde{\b\Pi}}
	\newcommand{\p}{\mathscr P}
	\newcommand{\pistol}{\mathparagraph}
  \newcommand{\pmax}{\mathbb{P}_{\text{max}}}
	\newcommand{\mw}{\mathcal{M}_{\textsf{mw}}^\sharp}
  \newcommand{\vto}[2]{\begin{pmatrix}#1\\#2\end{pmatrix}}
  \newcommand{\game}[8]{\eq{\begin{array}{ccccccccc} \text{I} & #1 && #3 && #5 && #7\\ \text{II} && #2 && #4 && #6 && #8 \end{array}}}
  \newcommand{\vtre}[3]{\begin{pmatrix}#1\\#2\\#3\end{pmatrix}}
  \newcommand{\mto}[4]{\begin{pmatrix} #1 & #2 \\ #3 & #4\end{pmatrix}}
  \newcommand{\mtre}[9]{\begin{pmatrix} #1 & #2 & #3 \\ #4 & #5 & #6 \\ #7 & #8 & #9\end{pmatrix}}
  \newcommand{\bra}[1]{\langle #1\rangle}
  \newcommand{\dbra}[1]{\llbracket #1 \rrbracket}
  \newcommand{\norm}[1]{\left|\left|#1\right|\right|}
  \newcommand{\abs}[1]{\left|#1\right|}
  \newcommand{\later}{{\vartriangleright}}
  \newcommand{\normal}{\unlhd}
  \newcommand{\ideal}{\unlhd}
  \newcommand{\rel}{\ \text{rel}\ }
  \newcommand{\pnormal}{\mathrel{\ooalign{$\lneq$\cr\raise.22ex\hbox{$\lhd$}\cr}}}
  \newcommand{\pideal}{\mathrel{\ooalign{$\lneq$\cr\raise.22ex\hbox{$\lhd$}\cr}}}
  \newcommand{\acts}{\curvearrowright}
  \newcommand{\colimm}{\varinjlim}
  \newcommand{\limm}{\varprojlim}
  \newcommand{\eff}{\mathcal{E}\! f\! f}
  \newcommand{\set}{\textsf{Set}}
  \newcommand{\fin}{\textsf{Fin }}
  \newcommand{\Top}{\textsf{Top}}
  \newcommand{\zf}{\textsf{ZF}}
  \newcommand{\zfc}{\textsf{ZFC}}
  \newcommand{\ch}{\textsf{CH}}
  \newcommand{\gch}{\textsf{GCH}}
  \newcommand{\ad}{\textsf{AD}}
  \newcommand{\ac}{\textsf{AC}}
  \newcommand{\dc}{\textsf{DC}}
  \newcommand{\cat}{\textsf{Cat}}
  \newcommand{\grp}{\textsf{Grp}}
  \newcommand{\shc}{\mathcal{SHC}}
  \newcommand{\sset}{\textsf{sSet}}
  \newcommand{\gset}{$G$\textsf{Set}}
  \newcommand{\ab}{\textsf{Ab}}
  \newcommand{\godel}[1]{\ulcorner #1 \urcorner}
  \newcommand{\circled}[1]{\raisebox{-0.5pt}{\textcircled{\raisebox{-0.5pt} {{\scriptsize #1}}}}}
  \newcommand{\rmod}{{_R\textsf{Mod}}}
  \newcommand{\modr}{{\textsf{Mod}_R}}
  \newcommand{\lex}{<_{\text{lex}}}
  \newcommand{\po}{\ar@{}[dr]|{\text{\pigpenfont R}}}
  \newcommand{\pb}{\ar@{}[dr]|{\text{\pigpenfont J}}}
  \newcommand{\init}{\unlhd}	
  \newcommand{\pinit}{\lhd}	
