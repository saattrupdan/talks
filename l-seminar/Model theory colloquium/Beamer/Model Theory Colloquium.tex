\documentclass{beamer}
\usetheme{Warsaw}
\setbeamercovered{transparent}
\usepackage[utf8x]{inputenc}
\DeclareMathOperator{\proves}{\vdash}

\title[Model Theory]{Model Theory \\ An introduction}
\author{Dan Saattrup Nielsen}
\date{October 25, 2013}

\begin{document}

\begin{frame}
	\titlepage
\end{frame}

% TOC
\begin{frame}{Contents}
	\begin{itemize}
		\item<1-4> Introduction
		\item<2-3> Basic concepts
		\item<3> Awesome theorems
	\end{itemize}
\end{frame}


\begin{frame}{So.. what \textit{is} model theory?}
\begin{itemize}
\pause\item Model theory is about \textit{models} and \textit{theories}
\pause\item Classical model theory = algebra + logic
\pause\item Syntax and semantics
\end{itemize}
\end{frame}


\begin{frame}{Contents}
	\begin{itemize}
		\item<0> Introduction
		\item<1> Basic concepts
		\item<0> Awesome theorems
	\end{itemize}
\end{frame}


\begin{frame}{Languages}
\begin{block}{Definition}
A \textit{language} $\mathcal{L}$\pause\ is a set of relation symbols $(R_i)_{i\in I}$\pause, function symbols $(f_j)_{j\in J}$\pause\ and constant symbols $(c_k)_{k\in K}$.
\end{block}
\end{frame}


\begin{frame}{Structures}
\begin{block}{Definition}
	Let $\mathcal{L}$ be a language.\pause\ Then an $\mathcal{L}$-\textit{structure} $\mathfrak{M}$ is a set $M$ along with
\begin{itemize}
	\pause\item an n-ary relation $R_i^\mathfrak{M}\subseteq M^n$ for each n-ary relation symbol $R_i\in\mathcal{L}$,
	\pause\item an n-ary function $f_j^\mathfrak{M}:M^n\to M$ for each n-ary function symbol $f_j\in\mathcal{L}$
	\pause\item a distinguished element $c_k^\mathfrak{M}\in M$ for each constant symbol $c_k\in\mathcal{L}$.
\end{itemize}				
\end{block}
\end{frame}


\begin{frame}{Theories}
Let $\mathcal{L}$ be a language.
\pause

\begin{block}{Definition}
An $\mathcal{L}$-\textit{theory} $\mathcal{T}$ is just a set of $\mathcal{L}$-sentences.\pause\ $\mathcal{T}$ is called \textit{satisfiable} if it has a model.
\end{block}
\end{frame}


\begin{frame}{Contents}
	\begin{itemize}
		\item<0> Introduction
		\item<0> Basic concepts
		\item<1> Awesome theorems
	\end{itemize}
\end{frame}


\begin{frame}{Completeness (first kind)}
	\pause	
	\begin{block}{Definition}
		A \textit{dense linear ordering} $(M,<)$ is an ordered set on which the ordering $<$ is
		\begin{itemize}
			\pause\item irreflexive
			\pause\item transitive
			\pause\item total
			\pause\item dense.
		\end{itemize}
	\end{block}
\end{frame}


\begin{frame}{Completeness (first kind)}	
	Let $\mathcal{L}_{strict}:=\{<\}$ be the language of strict orderings.\\
	
	\pause
	
	\begin{block}{Proposition}
		Let $\mathfrak{M}$, $\mathfrak{N}$ be two DLO's without endpoints\pause\ and let $\sigma$ be an $\mathcal{L}_{strict}$-sentence.\pause\ Then
		\begin{align*}
			\mathfrak{M}\models\sigma\Leftrightarrow\mathfrak{N}\models\sigma.
		\end{align*}
	\end{block}
	
	\pause	
	
	\begin{block}{Proposition}
		Let $\mathfrak{M}$ be a DLO without endpoints.\pause\ There exists no $\mathcal{L}_{strict}$-sentence $\sigma$ such that $\mathfrak{M}\models\sigma$\pause\ iff the ordering on $\mathfrak{M}$ is complete.
	\end{block}
\end{frame}


\begin{frame}{Completeness (second kind)}
	Let $\mathcal{L}$ be a language.\\
	\pause
		
	\begin{block}{Definition}
		A theory $\mathcal{T}$ is \textit{complete} if for every $\mathcal{L}$-sentence $\sigma$,\pause\ either $\mathcal{T}\models\sigma$ or $\mathcal{T}\models\lnot\sigma$.
	\end{block}
	
	\pause
	
	\begin{block}{Theorem (Vaught's test)}
		Let $\mathcal{T}$ be a satisfiable $\mathcal{L}$-theory with no finite models\pause\ and every model of cardinality $\kappa$ is isomorphic, for some cardinal $\kappa\geq|\mathcal{L}|$.\pause\ Then $\mathcal{T}$ is complete.
	\end{block}
	
	\pause
	
	\begin{block}{Corollary}
		\textsf{Vec}$_\infty$ is complete in $\mathcal{L}_{vec}$, the language of vector spaces.
	\end{block}
\end{frame}


\begin{frame}{Completeness (third kind)}
	Let $\mathcal{L}$ be a language and $\mathcal{T}$ an $\mathcal{L}$-theory.\\
	
	\pause
	
	\begin{block}{Theorem (Gödels Completeness Theorem)}
		Let $\sigma$ be an $\mathcal{L}$-sentence.\pause\ Then
		\begin{align*}
			\mathcal{T}\models\sigma\Leftrightarrow\mathcal{T}\proves\sigma.
		\end{align*}
	\end{block}
	
	\pause
	
	\begin{block}{Theorem (Compactness Theorem)}
		$\mathcal{T}$ is satisfiable iff every finite $\Delta\subseteq\mathcal{T}$ is satisfiable.
	\end{block}
	
	\pause
	
	\begin{block}{Corollary}
		If $\mathcal{T}\models\sigma$ then $\Delta\models\sigma$ for some finite $\Delta\subseteq\mathcal{T}$.
	\end{block}
\end{frame}


\begin{frame}{Completeness (third kind)}
	\begin{block}{Lemma}
		Let $\sigma$ be a sentence in the language of rings.\pause\ Then \textsf{ACF}$_0\models\sigma$ if \textsf{ACF}$_p\models\sigma$ for all primes $p$.\pause\ In particular $\mathbb{C}\models\sigma$.
	\end{block}
	
	\pause	
	
	\begin{block}{Theorem (Ax)}
		Every injective polynomial $p:\mathbb{C}^n\to\mathbb{C}^n$ is surjective.
	\end{block}
\end{frame}

\end{document}