% Document definition
\documentclass{beamer}
\usetheme[block=fill]{metropolis}

% Packages
% General packages
  \usepackage[light]{antpolt} % Provides awesome font
  \usepackage{hyperref} % Provides \url and clickable links in the pdf

  % Math packages
  \usepackage{amsmath, amssymb, amsfonts} % Math symbolic jargon
  \usepackage{amsthm} % Theorem environment
  \usepackage{mathrsfs}	% Provides the \mathscr{} curly font
  \usepackage{stmaryrd}	% Provides the \lightning symbol

  % Diagrams
  \usepackage{tikz} % Awesome diagrams
    \usetikzlibrary{cd} % Commutative diagrams 
  \usepackage[all]{xy} % Provides xymatrix environment for diagrams
		
% User-defined commands
  % General things	
  \newcommand{\eq}[1]{\begin{align*} #1 \end{align*}}
  \newcommand{\eqq}[1]{\begin{align*} #1\\ \end{align*}}
  \newcommand{\pix}[2][1]{\begin{center}\includegraphics[scale=#1]{#2}\\\end{center}}
  \newcommand{\cd}[1]{\eq{\xymatrix{#1}}}

  % Declared operators
  \DeclareMathOperator{\dirlim}{dirlim}
  \DeclareMathOperator{\iterates}{I}
  \DeclareMathOperator{\code}{Code}
  \DeclareMathOperator{\piterates}{pI}
  \DeclareMathOperator{\blowups}{B}
  \DeclareMathOperator{\pblowups}{pB}
  \DeclareMathOperator{\lhod}{\triangleleft_{\mathrm{HOD}}}
  \DeclareMathOperator{\lehod}{\trianglelefteq_{\mathrm{HOD}}}
  \DeclareMathOperator{\ledj}{\le_{\mathrm{DJ}}}
  \DeclareMathOperator{\nledj}{\not{\le}_{\mathrm{DJ}}}
  \DeclareMathOperator{\ldj}{<_{\mathrm{DJ}}}
  \DeclareMathOperator{\di}{\textsf{DI}}
  \DeclareMathOperator{\hc}{\textsf{HC}}
  \DeclareMathOperator{\M}{\mathcal M}
  \DeclareMathOperator{\N}{\mathcal N}
  \DeclareMathOperator{\K}{\mathcal K}
  \DeclareMathOperator{\F}{\mathcal F}
  \DeclareMathOperator{\Q}{\mathcal Q}
  \DeclareMathOperator{\W}{\mathcal W}
  \DeclareMathOperator{\V}{\mathcal V}
  \DeclareMathOperator{\T}{\mathcal T}
  \DeclareMathOperator{\U}{\mathcal U}
  \DeclareMathOperator{\B}{\mathcal B}
  \DeclareMathOperator{\R}{\mathcal R}
  \DeclareMathOperator{\D}{\mathcal D}
  \DeclareMathOperator{\G}{\mathcal G}
  \DeclareMathOperator{\C}{\mathcal C}
  \DeclareMathOperator{\A}{\mathcal A}
  \DeclareMathOperator{\h}{\mathcal H}
  \DeclareMathOperator{\I}{\mathcal I}
  \DeclareMathOperator{\core}{\mathfrak C}
  \DeclareMathOperator{\mc}{\textsf{MC}}
  \DeclareMathOperator{\refl}{Refl}
  \DeclareMathOperator{\pp}{pp}
  \DeclareMathOperator{\rk}{rk}
  \DeclareMathOperator{\otp}{otp}
  \DeclareMathOperator{\pr}{pr}
  \DeclareMathOperator{\lp}{Lp}
  \DeclareMathOperator{\In}{in}
  \DeclareMathOperator{\sgn}{sgn}
  \DeclareMathOperator{\lcm}{lcm}
  \DeclareMathOperator{\ran}{ran}
  \DeclareMathOperator{\cod}{cod}
  \DeclareMathOperator{\dom}{dom}	
  \DeclareMathOperator{\cond}{cond}
  \DeclareMathOperator{\con}{Con}
  \DeclareMathOperator{\rank}{rank}
  \DeclareMathOperator{\gal}{Gal}
  \DeclareMathOperator{\cov}{Cov}
  \DeclareMathOperator{\im}{im}
  \DeclareMathOperator{\add}{Add}
  \DeclareMathOperator{\supp}{supp}
  \DeclareMathOperator{\sub}{Sub}
  \DeclareMathOperator{\diam}{diam}
  \DeclareMathOperator{\hod}{HOD}
  \DeclareMathOperator{\od}{OD}
  \DeclareMathOperator{\codom}{codom}
  \DeclareMathOperator{\Det}{Det}
  \DeclareMathOperator{\len}{len}
  \DeclareMathOperator{\ma}{MA}
  \DeclareMathOperator{\id}{id}
  \DeclareMathOperator{\sing}{Sing}
  \DeclareMathOperator{\pred}{pred}
  \DeclareMathOperator{\cl}{cl}
  \DeclareMathOperator{\Int}{int}
  \DeclareMathOperator{\ob}{Ob}
  \DeclareMathOperator{\mor}{Mor}
  \DeclareMathOperator{\sh}{Sh}
  \DeclareMathOperator{\ot}{ot}
  \DeclareMathOperator{\ult}{Ult}
  \DeclareMathOperator{\sg}{sg}
  \DeclareMathOperator{\env}{Env}
  \DeclareMathOperator{\tor}{Tor}
  \DeclareMathOperator{\ext}{Ext}
  \DeclareMathOperator{\comp}{\textsf{Comp}}
  \DeclareMathOperator{\card}{card}
  \DeclareMathOperator{\Card}{Card}
  \DeclareMathOperator{\cf}{cf}
  \DeclareMathOperator{\cof}{cof}
  \DeclareMathOperator{\cc}{\text{\textasciicircum}}
  \DeclareMathOperator{\sk}{sk}
  \DeclareMathOperator{\crit}{crit}
  \DeclareMathOperator{\cls}{cls}
  \DeclareMathOperator{\pd}{pd}
  \DeclareMathOperator{\ev}{ev}
  \DeclareMathOperator{\wo}{WO}
  \DeclareMathOperator{\wfp}{wfp}
  \DeclareMathOperator{\xor}{\oplus}
  \DeclareMathOperator{\nor}{\downarrow}
  \DeclareMathOperator{\nand}{\uparrow}
  \DeclareMathOperator{\biglor}{\bigvee}
  \DeclareMathOperator{\bigland}{\bigwedge}
  \DeclareMathOperator{\Lr}{\Leftrightarrow}
  \DeclareMathOperator{\lr}{\leftrightarrow}
  \DeclareMathOperator{\ip}{\perp\!\!\!\perp}
  \DeclareMathOperator{\psubset}{\subsetneq}
  \DeclareMathOperator{\psupset}{\supsetneq}
  \DeclareMathOperator{\elsub}{\preceq}
  \DeclareMathOperator{\elsup}{\succeq}
  \DeclareMathOperator{\pelsub}{\prec}
  \DeclareMathOperator{\pelsup}{\succ}
  \DeclareMathOperator{\contr}{\lightning}
  \DeclareMathOperator{\proves}{\vdash}
  \DeclareMathOperator{\nproves}{\nvdash}
  \DeclareMathOperator{\nmodels}{\nvDash}
  \DeclareMathOperator{\forces}{\Vdash}
  \DeclareMathOperator{\nforces}{\nVdash}
  \DeclareMathOperator{\adj}{\dashv}
  \DeclareMathOperator{\restr}{\upharpoonright}
  \DeclareMathOperator{\ex}{\underline{ex}}
  \DeclareMathOperator{\st}{\underline{st}}
  \DeclareMathOperator{\sv}{\underline{sv}}
  \DeclareMathOperator{\tl}{\underline{tl}}
  \DeclareMathOperator{\tensor}{\otimes}
  \DeclareMathOperator{\monus}{\dotdiv}
  \DeclareMathOperator{\Null}{\textsf{NULL}}
  \DeclareMathOperator{\nat}{Nat}
  \DeclareMathOperator{\col}{Col}
  \DeclareMathOperator{\Root}{root}
  \DeclareMathOperator{\aut}{Aut}
  \DeclareMathOperator{\spec}{spec}
  \DeclareMathOperator{\sq}{Sq}
  \DeclareMathOperator{\ann}{ann}
  \DeclareMathOperator{\proj}{proj}
  \DeclareMathOperator{\ffrac}{Frac}
  \DeclareMathOperator{\hull}{Hull}
  \DeclareMathOperator{\chull}{cHull}
  \DeclareMathOperator{\lh}{lh}
  \DeclareMathOperator{\Def}{Def}
  \DeclareMathOperator{\Span}{span}
  \DeclareMathOperator{\form}{Form}
  \DeclareMathOperator{\tc}{trcl}


  % Redeclared operators
  \renewcommand{\B}{\mathbb B}
  \renewcommand{\P}{\mathcal P}
  \renewcommand{\S}{\mathcal S}
	\renewcommand{\succ}{\text{succ}}
	\renewcommand{\pr}{\text{Pr}}
  \renewcommand{\subset}{\subseteq}
  \renewcommand{\supset}{\supseteq}
  \renewcommand{\hom}{\text{Hom}}
	\renewcommand{\index}{\text{index }}
  \renewcommand{\a}{\b{a}}
	\renewcommand{\r}{{^\omega\omega}}
	\renewcommand{\l}{|}

  % Convenient shortcuts
	\newcommand{\E}{\vec E}
	\newcommand{\bSigma}{\utilde{\b\Sigma}}
	\newcommand{\bPi}{\utilde{\b\Pi}}
	\newcommand{\p}{\mathscr P}
	\newcommand{\pistol}{\mathparagraph}
  \newcommand{\pmax}{\mathbb{P}_{\text{max}}}
	\newcommand{\mw}{\mathcal{M}_{\textsf{mw}}^\sharp}
  \newcommand{\vto}[2]{\begin{pmatrix}#1\\#2\end{pmatrix}}
  \newcommand{\game}[8]{\eq{\begin{array}{ccccccccc} \text{I} & #1 && #3 && #5 && #7\\ \text{II} && #2 && #4 && #6 && #8 \end{array}}}
  \newcommand{\vtre}[3]{\begin{pmatrix}#1\\#2\\#3\end{pmatrix}}
  \newcommand{\mto}[4]{\begin{pmatrix} #1 & #2 \\ #3 & #4\end{pmatrix}}
  \newcommand{\mtre}[9]{\begin{pmatrix} #1 & #2 & #3 \\ #4 & #5 & #6 \\ #7 & #8 & #9\end{pmatrix}}
  \newcommand{\bra}[1]{\langle #1\rangle}
  \newcommand{\dbra}[1]{\llbracket #1 \rrbracket}
  \newcommand{\norm}[1]{\left|\left|#1\right|\right|}
  \newcommand{\abs}[1]{\left|#1\right|}
  \newcommand{\later}{{\vartriangleright}}
  \newcommand{\normal}{\unlhd}
  \newcommand{\ideal}{\unlhd}
  \newcommand{\rel}{\ \text{rel}\ }
  \newcommand{\pnormal}{\mathrel{\ooalign{$\lneq$\cr\raise.22ex\hbox{$\lhd$}\cr}}}
  \newcommand{\pideal}{\mathrel{\ooalign{$\lneq$\cr\raise.22ex\hbox{$\lhd$}\cr}}}
  \newcommand{\acts}{\curvearrowright}
  \newcommand{\colimm}{\varinjlim}
  \newcommand{\limm}{\varprojlim}
  \newcommand{\eff}{\mathcal{E}\! f\! f}
  \newcommand{\set}{\textsf{Set}}
  \newcommand{\fin}{\textsf{Fin }}
  \newcommand{\Top}{\textsf{Top}}
  \newcommand{\zf}{\textsf{ZF}}
  \newcommand{\zfc}{\textsf{ZFC}}
  \newcommand{\vp}{\textsf{VP}}
  \newcommand{\gvp}{\textsf{gVP}}
  \newcommand{\on}{\textsf{On}}
  \newcommand{\wvp}{\textsf{WVP}}
  \newcommand{\gwvp}{\textsf{gWVP}}
  \newcommand{\gbc}{\textsf{GBC}}
  \newcommand{\km}{\textsf{KM}}
  \newcommand{\ch}{\textsf{CH}}
  \newcommand{\gch}{\textsf{GCH}}
  \newcommand{\ad}{\textsf{AD}}
  \newcommand{\ac}{\textsf{AC}}
  \newcommand{\dc}{\textsf{DC}}
  \newcommand{\cat}{\textsf{Cat}}
  \newcommand{\grp}{\textsf{Grp}}
  \newcommand{\shc}{\mathcal{SHC}}
  \newcommand{\sset}{\textsf{sSet}}
  \newcommand{\gset}{$G$\textsf{Set}}
  \newcommand{\ab}{\textsf{Ab}}
  \newcommand{\godel}[1]{\ulcorner #1 \urcorner}
  \newcommand{\circled}[1]{\raisebox{-0.5pt}{\textcircled{\raisebox{-0.5pt} {{\scriptsize #1}}}}}
  \newcommand{\rmod}{{_R\textsf{Mod}}}
  \newcommand{\modr}{{\textsf{Mod}_R}}
  \newcommand{\lex}{<_{\text{lex}}}
  \newcommand{\po}{\ar@{}[dr]|{\text{\pigpenfont R}}}
  \newcommand{\pb}{\ar@{}[dr]|{\text{\pigpenfont J}}}
  \newcommand{\init}{\unlhd}	
  \newcommand{\pinit}{\lhd}	

\begin{document}

\title[Virtual Large Cardinals]{Virtual Large Cardinals\\ {\small\textsc{British Logic Colloquium 2019, Oxford}}}
\author[Dan Saattrup Nielsen]{Dan Saattrup Nielsen\\ University of Bristol}
\date{September 5, 2019}

\begin{frame}
	\titlepage
\end{frame}

\begin{frame}{What is a large cardinal?}
  There is no rigorous definition of a large cardinal\pause, but many of them, especially in the higher reaches of the hierarchy, are characterised as the \textit{critical point} of a certain elementary embedding:
  \eq{
    \pi\colon(\M,\in)\to(\N,\in)
  }

  \pause
  \begin{block}{Examples}
    \begin{itemize}
      \item $\kappa$ is a \textbf{measurable cardinal} if $\M=H_{\kappa^+}$.
      \item $\kappa$ is a \textbf{$\theta$-strong cardinal} if $\M=H_\theta$, $H_\theta\subset\N$ and $\pi(\kappa)>\theta$.
    \end{itemize}
  \end{block}

  Recall that $x\in H_\theta$ iff $\abs{\tc(x)} < \theta$. This hierarchy is often more convenient than the $V_\alpha$'s since $H_\theta\models\zfc^-$ if $\theta$ is regular.
\end{frame}

{\setbeamercolor{background canvas}{bg = white}
\begin{frame}{The hierarchy of large cardinals}
  \pix[0.65]{gfx/hierarchy.pdf}
\end{frame}}

\begin{frame}{What is a \emph{virtual} large cardinal?}
  We basically just require that the embeddings exist in a \emph{generic extension} rather than in $V$:

  \pause

  \begin{block}{"Definition"}
    Let $\Phi$ be a large cardinal concept defined via elementary embeddings between \textit{sets}, like the definitions on the previous slide.

    \pause

    Then $\kappa$ is \textbf{virtually $\Phi$} if the same definition holds but where we only require the embeddings exist in a generic extension and that $\N\subset V$.
  \end{block}
\end{frame}

{\setbeamercolor{background canvas}{bg = white}
\begin{frame}{A virtual addition to the hierarchy}
  \pix[0.65]{gfx/hierarchy_w_virtuals.pdf}
\end{frame}}

\begin{frame}{Attaching an adjective}
  \only<1-3>{
    Let us attach a \textbf{pre-} to our large cardinals if we do not require anything about where the critical point is sent:

    \begin{block}{Example}
      $\kappa$ is \textbf{prestrong} if for every regular $\theta>\kappa$ there is an elementary embedding $\pi\colon (H_\theta,\in)\to(\N,\in)$ with $\crit\pi=\kappa$ and $H_\theta\subset\N$.
    \end{block}
  }

  \only<2>{  
    This is really not an interesting concept in the real world:

    \begin{block}{Proposition (folklore)}
      For regular cardinals $\kappa<\theta$:
      \begin{itemize}
        \item $\kappa$ is $\theta$-prestrong iff it is $\theta$-strong
        \item $\kappa$ is $\theta$-presupercompact iff it is $\theta$-supercompact
      \end{itemize}
    \end{block}
  }

  \only<3>{
    It \textit{is} interesting in the virtual world, however:

    \begin{block}{Theorem (N.)}
      For regular cardinals $\kappa<\theta$, $\kappa$ is virtually $\theta$-prestrong iff either
      \begin{itemize}
        \item $\kappa$ is virtually $\theta$-strong, or
        \item $\kappa$ is virtually $(\theta,\omega)$-superstrong
      \end{itemize}
    \end{block}
  }
\end{frame}

\begin{frame}{Characterising a phenomenon}
  \begin{block}{Corollary (N.)}
    Virtually $\theta$-prestrongs are equiconsistent with virtually $\theta$-strongs.
  \end{block}

  \pause

  \begin{block}{Corollary (N.)}
    The following are equivalent:
    \begin{itemize}
      \item Virtually prestrongs are equivalent to virtually strongs
      \item There are no virtually $\omega$-superstrongs.
    \end{itemize}
  \end{block}

  \pause

  Note that $\omega$-superstrong cardinals are inconsistent with $\zfc$!
\end{frame}

\begin{frame}{Adding parameters}
  \begin{block}{Definition}
    Let $\kappa<\theta$ be regular and let $A$ be a class. Then $\kappa$ is \textbf{virtually $(\theta,A)$-prestrong} if there exists a generic elementary embedding
    \eq{
      \pi\colon(H_\theta^V,\in,A\cap H_\theta^V)\to(\N,\in,B)
    }

    such that $\crit\pi=\kappa$, $H_\theta^V\subset\N$, $\N\subset V$ and $A\cap H_\theta^V=B\cap H_\theta^V$.

    \qquad Further, if $\pi(\kappa)>\theta$ then $\kappa$ is \textbf{virtually $(\theta,A)$-strong}.
  \end{block}

  \pause
  
  Can we find some virtual large cardinal characterising exactly when the virtually $A$-prestrongs are equivalent to virtually $A$-strongs?

  \pause

  Remember that we are looking for a large cardinal notion which is inconsistent in the real world.
\end{frame}

\begin{frame}{Berkeley cardinals}
  \begin{block}{Definition}
    $\delta$ is \textbf{virtually berkeley} if for every transitive set $\M$ there exists a generic elementary embedding $\pi\colon\M\to\M$ with $\crit\pi<\delta$.
  \end{block}

  \only<2>{The real world versions of these are of course inconsistent with $\zfc$, but are currently being investigated in a choiceless context.}

  \only<3->{
    \begin{block}{Definition}
      Say that $\on$ is \textbf{virtually (pre)woodin} if for every class $A$ there exists a virtually $A$-(pre)strong cardinal $\kappa$.
    \end{block}
  }

  \onslide<4->{
    \begin{block}{Theorem (N.)}
      The following are equivalent:
      \begin{itemize}
        \item $\on$ is virtually prewoodin iff it is virtually woodin
        \item There are no virtually berkeley cardinals
      \end{itemize}
    \end{block}
  }
\end{frame}

\begin{frame}{Open questions}
  \begin{block}{Question}
    Is the existence of a virtually berkeley cardinal equivalent to the statement that, for every class $A$, every virtually $A$-prestrong cardinal is virtually $A$-strong?
  \end{block}

  \begin{block}{Question}
    Are virtually $\omega$-superstrongs equivalent to virtually berkeleys?

    Wilson (2018) has shown that this is true in $L$.
  \end{block}
\end{frame}

{\setbeamercolor{background canvas}{bg = white}
\begin{frame}
  \phantom{hidden text}
  \begin{center}
    {\LARGE Thank you for your attention.}
  \end{center}

  \pix[0.65]{gfx/virtual_hierarchy.pdf}
\end{frame}}

\end{document}
