\documentclass{beamer}
\usetheme{Singapore}
\setbeamertemplate{navigation symbols}{}

% Packages
  % General packages
  \usepackage[utf8x]{inputenc} % Danish language support
  \usepackage[light]{antpolt} % Provides awesome font
  \usepackage{hyperref} % Provides \url and clickable links in the pdf
  \usepackage{cleveref}	% Provides \cref and \Cref for including text before references
  \usepackage{graphicx}	% Provides the \includegraphics[]{} command
  \usepackage{comment} % Provides comment-environment for multi-line commenting
  \usepackage{twoopt} % Allows adding commands with two optional arguments
  \usepackage{setspace} % Provides onehalfspacing environment
  \usepackage{enumitem}	% Provides control of spacing in lists
  \usepackage{amsbsy} % Provides \boldsymbol command
  \usepackage{lipsum} % Provides \lipsum command
  \usepackage{apacite} % Provides bibliography style
	
  % Math packages
  \usepackage{amsmath, amssymb, amsfonts} % Math symbolic jargon
  \usepackage{amsthm} % Theorem environment
  \usepackage{mathrsfs}	% Provides the \mathscr{} curly font
  \usepackage{stmaryrd}	% Provides the \lightning symbol and semantics-brackets, among others
  \usepackage{tikz} % Awesome diagrams
  \usepackage{tikz-cd} % For making tree cds
    \usetikzlibrary{matrix,arrows} % Matrices and arrows for style points
    \usetikzlibrary{decorations.pathreplacing,calc,arrows.meta} % For making tree cds
  %\usepackage[all]{xy} % Provides xymatrix environment for diagrams

  % Beamer packages
  \usepackage[absolute,overlay]{textpos}  % Placement of textblocks
  \usepackage{mathdots}  % Provides \iddots command

% Tree diagrams
\tikzset{
  tree/.style 2 args={
    decorate,
    decoration={
      show path construction,
      lineto code={
        \draw[dotted,-] (\tikzinputsegmentfirst) --($(\tikzinputsegmentfirst)!.5!(\tikzinputsegmentlast)$);
        \draw[-{Latex}] ($(\tikzinputsegmentfirst)!.5!(\tikzinputsegmentlast)$) --(\tikzinputsegmentlast) node [midway,right] {\small{$#1$}};
        \draw[-] (\tikzinputsegmentfirst) --++ (105:0.65cm);
        \draw[-] (\tikzinputsegmentfirst) --++ (75:0.65cm) node [midway, right] {\small{$#2$}};
      }
    }
  }
}

% User-defined commands
  % General things	
  \newcommand{\eq}[1]{\begin{align*} #1 \end{align*}}
  \newcommand{\eqq}[1]{\begin{align*} #1\\ \end{align*}}
  \newcommand{\pic}[1]{\begin{center}\includegraphics[scale=.5]{#1}\\\end{center}}
  \newcommand{\pix}[2]{\begin{center}\includegraphics[scale=#2]{#1}\\\end{center}}
  %\newcommand{\cd}[1]{\eq{\xymatrix@L=6pt{#1}}}
  \renewcommand{\b}[1]{{\bf #1}}

  % Declared operators
  \DeclareMathOperator{\rud}{rud}
  \DeclareMathOperator{\pred}{pred}
  \DeclareMathOperator{\card}{card}
  \DeclareMathOperator{\on}{On}
  \DeclareMathOperator{\val}{val}
  \DeclareMathOperator{\ran}{ran}
  \DeclareMathOperator{\cod}{cod}
  \DeclareMathOperator{\trcl}{trcl}
  \DeclareMathOperator{\dom}{dom}	
  \DeclareMathOperator{\rank}{rank}
  \DeclareMathOperator{\im}{im}
  \DeclareMathOperator{\ma}{MA}
  \DeclareMathOperator{\hull}{Hull}
  \DeclareMathOperator{\chull}{cHull}
  \DeclareMathOperator{\id}{id}
  \DeclareMathOperator{\cof}{cof}
  \DeclareMathOperator{\Th}{Th}
  \DeclareMathOperator{\sing}{Sing}
  \DeclareMathOperator{\cl}{cl}
  \DeclareMathOperator{\Int}{int}
  \DeclareMathOperator{\ob}{Ob}
  \DeclareMathOperator{\col}{Col}
  \DeclareMathOperator{\sat}{Sat}
  \DeclareMathOperator{\lh}{lh}
  \DeclareMathOperator{\mor}{Mor}
  \DeclareMathOperator{\ult}{Ult}
  \DeclareMathOperator{\comp}{\textsf{Comp}}
  \DeclareMathOperator{\zf}{\mathsf{ZF}}
  \DeclareMathOperator{\zfc}{\mathsf{ZFC}}
  \DeclareMathOperator{\ch}{\mathsf{CH}}
  \DeclareMathOperator{\gch}{\mathsf{GCH}}
  \DeclareMathOperator{\con}{Con}
  \DeclareMathOperator{\cf}{cf}
  \DeclareMathOperator{\crit}{crit}
  \DeclareMathOperator{\pd}{pd}
  \DeclareMathOperator{\ad}{\mathsf{AD}}
  \DeclareMathOperator{\ac}{\mathsf{AC}}
  \DeclareMathOperator{\xor}{\oplus}
  \DeclareMathOperator{\nor}{\downarrow}
  \DeclareMathOperator{\nand}{\uparrow}
  \DeclareMathOperator{\biglor}{\bigvee}
  \DeclareMathOperator{\bigland}{\bigwedge}
  \DeclareMathOperator{\Lr}{\Leftrightarrow}
  \DeclareMathOperator{\lr}{\leftrightarrow}
  \DeclareMathOperator{\ip}{\perp\!\!\!\perp}
  \DeclareMathOperator{\psubset}{\subsetneq}
  \DeclareMathOperator{\psupset}{\supsetneq}
  \DeclareMathOperator{\elsub}{\prec}
  \DeclareMathOperator{\elsup}{\succ}
  \DeclareMathOperator{\contr}{\lightning}
  \DeclareMathOperator{\proves}{\vdash}
  \DeclareMathOperator{\nproves}{\nvdash}
  \DeclareMathOperator{\nmodels}{\nvDash}
  \DeclareMathOperator{\forces}{\Vdash}
  \DeclareMathOperator{\nforces}{\nVdash}
  \DeclareMathOperator{\adj}{\dashv}
  \DeclareMathOperator{\restr}{\upharpoonright}
  \DeclareMathOperator{\ex}{\underline{ex}}
  \DeclareMathOperator{\st}{\underline{st}}
  \DeclareMathOperator{\sv}{\underline{sv}}
  \DeclareMathOperator{\tl}{\underline{tl}}
  \DeclareMathOperator{\tensor}{\otimes}
  \DeclareMathOperator{\M}{\mathcal{M}}
  \DeclareMathOperator{\N}{\mathcal{N}}
  \DeclareMathOperator{\Q}{\mathcal{Q}}
  \DeclareMathOperator{\R}{\mathcal{R}}
  \DeclareMathOperator{\W}{\mathcal{W}}
  \DeclareMathOperator{\F}{\mathcal{F}}
  \DeclareMathOperator{\T}{\mathcal{T}}
  \DeclareMathOperator{\U}{\mathcal{U}}
  \DeclareMathOperator{\V}{\mathcal{V}}
  \DeclareMathOperator{\G}{\mathcal{G}}
  \DeclareMathOperator{\A}{\mathcal{A}}
  \DeclareMathOperator{\pr}{pr}
  \DeclareMathOperator{\Root}{root}
  \DeclareMathOperator{\wfp}{wfp}
  \DeclareMathOperator{\Def}{Def}

  % Redeclared operators
  \renewcommand{\subset}{\subseteq}
  \renewcommand{\supset}{\supseteq}
  \newcommand{\nsubset}{\nsubseteq}
  \newcommand{\nsupset}{\nsupseteq}
  \renewcommand{\hom}{\text{Hom}}
  \renewcommand{\P}{\mathcal{P}}
  \renewcommand{\S}{\mathcal{S}}
  \renewcommand{\H}{\mathcal{H}}
  \renewcommand{\succ}{\text{succ}}

  % Convenient shortcuts
  \newcommand{\vto}[2]{\begin{pmatrix}#1\\#2\end{pmatrix}}
  \newcommand{\vtre}[3]{\begin{pmatrix}#1\\#2\\#3\end{pmatrix}}
  \newcommand{\mto}[4]{\begin{pmatrix} #1 & #2 \\ #3 & #4\end{pmatrix}}
  \newcommand{\mtre}[9]{\begin{pmatrix} #1 & #2 & #3 \\ #4 & #5 & #6 \\ #7 & #8 & #9\end{pmatrix}}
  \newcommand{\bra}[1]{\langle #1\rangle}
  \newcommand{\dbra}[1]{\llbracket #1 \rrbracket}
  \newcommand{\norm}[1]{\left|\left|#1\right|\right|}
  \newcommand{\abs}[1]{\left|#1\right|}
  \newcommand{\normal}{\unlhd}
  \newcommand{\ideal}{\unlhd}
  \newcommand{\init}{\unlhd}
  \newcommand{\core}{\mathfrak C}
  \newcommand{\E}{\vec{E}}
  \newcommand{\J}{\mathcal{J}}
  \newcommand{\rel}{\ \text{rel}\ }
  \newcommand{\pnormal}{\mathrel{\ooalign{$\lneq$\cr\raise.22ex\hbox{$\lhd$}\cr}}}
  \newcommand{\pideal}{\mathrel{\ooalign{$\lneq$\cr\raise.22ex\hbox{$\lhd$}\cr}}}
  \newcommand{\pinit}{\lhd}
  \newcommand{\acts}{\curvearrowright}
  \newcommand{\colimm}{\varinjlim}
  \newcommand{\limm}{\varprojlim}
  \newcommand{\set}{\textsf{Set}}
  \newcommand{\godel}[1]{\ulcorner #1 \urcorner}
  \newcommand{\game}[8]{\eq{\begin{array}{ccccccccc} \text{I} & #1 && #3 && #5 && #7\\ \text{II} && #2 && #4 && #6 && #8 \end{array}}}
  \newcommand{\bgame}[8]{\eq{\begin{array}{ccccccccc} \text{I} & #1 & #3 & #5 & #7 \\ \text{II} & #2 & #4 & #6 & #8 \end{array}}}
  \newcommand{\los}{{{\fontfamily{arial}\selectfont\L}o\' s}}
